\chapter{威胁建模设计方法NEW}
\label{ch4}
\section{NEW 威胁建模概述}
随着云、容器和 API(Application Programming Interface,应用程序编程接口)
技术的快速发展,软件设计方法不断更替,敏捷的自动化工具、第三方库和各种
框架百花齐放。接口和应用程序面临的威胁在时刻变化中,传统的一些威胁已经
烟消云散,一些传统的建模方法应用在一些新兴领域,例如车联网,可能就变得
较为复杂和困难。在此基础上,我们设计了一种新的建模方法,
针对传统攻击树主观性强、叶节点概率难以确定等缺
点,提出了一种改进方法基于STRIDE和 FAHP。STRIDE 模型可以建立威胁和安全属
性之间的直接映射。它支持更好地理解和列出 TOE 的威
胁,而不是考虑与资产相关的攻击的无限可能性。FAHP 可
以计算出影响攻击成功概率的不同因素的权重。该方法如
图 4 所示。
\section{系统资产分析}
系统资产分析是安全威胁分析的第一步,主要是对 TOE 的
资产进行识别和分类。资产是需要保护的目标。参考汽车
行业的 EVITA 项目,车载网络的系统资产由车载设备、车
载设备上运行的应用以及各种 ECU 之间的通信链路组成
[12]。
\section{基于STRIDE的攻击树建模}
如图 5 所示,我们根据 STRIDE 关键字修改了攻击树。值
得注意的是,这里的攻击资产目标包含两种情况:一种
是高层次、抽象的资产目标,如我们提出的网络模型的
三个层次,另一种是具体的资产目标实体,如 CAN、ECU
等。确定系统的资产目标后,根据 STRIDE 关键字定义
的六类威胁进行威胁识别。我们并不试图重现 STRIDE
威胁建模的过程,而是使用其关键字来指导我们构建更
全面的攻击树,因此数据流图(DFD)在这里没有使用。通过这种方式,我们可以执
行完整的攻击树建模,并且不能忽略关键的安全威胁。
\newline
在图 5 中,攻击树中的根节点用 G 表示,子节点可
以分为两种:攻击资产目标节点和威胁节点。它们可以
分别用 Ai(i = 1,2,…,n)和 Ti(i = 1,2,…,n)
来表示。叶子节点代表的攻击事件或方法称为原子攻
击,标记为 Ei(i = 1,2,…,n)。实现根节点攻击目
标的一系列原子攻击的组合被定义为攻击序列 Pi (i =
1,2,… , n) 。 例 如 , 图 5 中 有 三 个 攻 击 序
列:P1{E1},P2{E2,E3},P3{E4}。
\section{攻击序列的量化}
攻击树模型的结果受叶节点概率的影响,因此叶节点的
概率值直接影响威胁分析的准确性。由于攻击序列是代
表完整攻击的一系列攻击叶节点的组合,因此这里对攻
击序列进行了量化。计算攻击序列的概率。攻击的可能
性受多种因素影响。最近关于安全指标的可靠性的工作
[21,22]表明,缺乏准确清晰的安全指标定义会导致指
标的主观性和偏差。攻击序列概率的安全度量不能任意
确定。因此,我们使用HEAVENS中定义的安全指标,这是相
对客观的。在这里,安全指标也称为安全属性。这个属
性在决策域,不在安全域。我们为每个攻击序列分配四
个安全属性:专业技能、TOE 知识、机会窗口和设备。采
用多属性效用理论将前面讨论的属性转化为效用值,以
实现攻击目标。这计算攻击序列概率的等式如(1)所示。
\newline
P W i x U W x k U W k w U W w e Ue i i i i 
\newline
其中 I 表示任何攻击序列。Pi 代表攻击序列出现的概
率。xi 是专家。ki 代表关于脚趾的知识。wi 是机会之
窗,ei 代表装备。Wf 代表攻击难度的权重。Wx 是专业
知识的权重,Wk 是关于 TOE 的知识的权重,Ww 是机会之
窗的权重,We 是设备的权重。这四个权重之和为
1。Uxi 代表提出了专业知识的效用价值。Uki 是公用事业关于 TOE 的知识值,Uwi 代表机会窗口的效用值和 Uei 代表展示设备的实用价值。值得关注 权重向量 W 是针对每个攻击序列的 11 12 1n
而不是整个系统,因为攻击行为不同的攻击序列所代表的效果也不同论安全属性权重。我们不能把重量向量作为TOE的常数。
此外,我们可以分析 xi、ki、wi 和
ei 与 Uxi、Uki、Uwi 和 Uei 成反比。因此,对于计算的方便性,它们之间的关系取为 U(x) = 1/x。
这里,当计算攻击序列的概率时,涉及到四个安全
属性,因此这是必要的制定相应的评分标准对其进行评估。采用的评分标准如表 2 所示。
\newline
对于一个每个参数的详细解释,请参考[19]。
表 2 与HEAVeNS中定义
的表 4-4 有些不同。我们的方法涉及到反比例,每个参
数的值都不能为零,所以每个值都比原值加 1。
\section{ 基于 FAHP 的安全属性权重
的确定}
对于不同的攻击序列,专家意见的权重、关于 TOE 的知
识、机会窗口和设备。FAHP 是对定性问题进行定量分析
的一种简单而直观的方法。表 3 所示的 0.1-0.9 标度标
准用于定量描述每个属性的相对重要性。
在 FAHP 中,应该基于特定的标度准则,通过两两比
较元素来构造模糊判断矩阵。如果模糊判断矩阵不一
致,则应将其转换为模糊一致判断矩阵。最后,利用模
糊一致判断矩阵计算各元素相对重要性的权重。
根据表 3 可以得到模糊判断矩阵 R

模糊判断矩阵的一致性应按以下性质进行检验

如果矩阵 R 满足三个条件,则该矩阵是模糊一致判
断矩阵。如果不是,为了确保两个元素的相对重要性的
一致性,使用基于等式 4 的算术平均来调整矩阵的各个
元素。公式如下:

r 是调整后的模糊一致判断矩阵 Ru 的元素。然后,n
是矩阵的阶。每个安全属性的权重向量 Wi 可以通过最
小二乘法对矩阵 Ru 进行归一化来计算。方程式如下
[24]:
\section{计算攻击序列的概率并进行风
险评估}

攻击序列代表一组攻击行为。它出现的概率表示在每个
攻击场景中攻击目标的可能性。当攻击序列所代表的攻
击行为实现时,一个攻击事件就完成了。我们可以根据
等式 1 计算攻击序列的概率,进行 TARA 来发现安全关键
系统中的威胁和漏洞,然后部署相应的安全防御机制。
\newline
基于提出的三层车载网络模型,可以先确定三个高级的资产对象类别,然后根据 EVITA
的资产定义导出具体的攻击资产。在 STRIDE 关键字的帮
助下识别安全威胁全面攻击树。由于车载网络的整个攻击树很难追踪,我
们以网络通信层的 CAN/CAN FD 攻击为例来演示所提出
的方法。枚举 西 W,西, ,W T 情商。(8)
所有相关的攻击都是复杂和不必要的。只有一些
1 2 3 n
这里显示了基于物理访问攻击的常见场景,以说明所提出
的方法的有效性
如图 6 所示。表 4 显示了攻击树中每个节点的含
义。
我们使用表 3 中的标度标准对每个攻击序列的安全
属性进行评分。为了证明该方法的有效性和避免个人评
价的主观性,我们进行了问卷调查,并邀请了十位专
家,包括相关研究领域的医生和大学教师。参与者的经
验越多在塔拉,结果就越准确可靠。最后,取其结果的平均值
并取整。评分结果如表 5 所示。
图 6 显示了五个攻击序列可以攻击 CAN/ CAN FD。然
后,用 FAHP 来判断和比较每个攻击序列的安全属性权
重,得到它们的模糊判断矩阵如下