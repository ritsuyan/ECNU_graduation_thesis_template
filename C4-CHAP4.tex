\chapter{基于STRIDE攻击树和FAHP的威胁建模方法}
\label{ch4}
本文提出了一种基于STRIDE和攻击树的安全威胁分析方法。首先介绍了三层车载网络模型,并介绍了该方法的定义和相关步骤,最后详细的介绍了通过FAHP进行安全属性的权重确定和攻击序列的概率确定。
\section{三层车载网络模型}
为了更直观的描述车载安全威胁,我们提出了三层车载网络模型,如图4.1所示。
\begin{figure}
  \centering
  \includegraphics[scale=0.5]{resources/img/c4.jpg}
  \caption{三层车载网络模型图}
\end{figure}
基于该模型,可以分析车载网络的资产目标,有利
于攻击树构建过程中威胁的描述。每一层的含义如下:
\begin{itemize}
  \item 终端节点层: 这一层包含车辆中各个域的 ECU 节点、
  传感器和执行器。这是模型的中心部分。如果受到攻
  击,它会直接影响车辆的安全。
  \item 网络通信层:该层由各种车载网络通信协议组成,如
  以太网、CAN/CAN FD、LIN 等。这一层的主要目的是传
  输数据并与之交互。
  \item 接口设备层:这一层包括各种可以与外部环境交互的通
  信设备接口,如 OBD、USB 等。
\end{itemize}
\begin{figure}
  \centering
  \includegraphics[scale=0.5]{resources/img/a13.png}
  \caption{SATT安全威胁建模步骤流程图}
\end{figure}
\section{SATT 安全威胁建模步骤概述}
我们已经从前面的章节知道STRIDE模型可以建立威胁和安全属
性之间的直接映射。它支持更好地理解和列出 TOE 的威
胁,而不是考虑与资产相关的攻击的无限可能性。FAHP 可
以计算出影响攻击成功概率的不同因素的权重。综合上述两个方法的优势和特点我们提出了SATT安全威胁分析方法,该方法具体步骤如
图 4.2 所示。
\subsection{系统资产分析}
系统资产分析是安全威胁分析的第一步,主要是对 TOE 的
资产进行识别和分类。资产是需要保护的目标。参考汽车
行业的 EVITA\cite{jager2015evita} 项目,EVITA架构的核心是车载网络架构的设计,验证和原型化。
在该架构中,可以保护与网络安全相关的组件免受篡改,并保护敏感数据免受
危害。车载网络的系统资产由车载设备、车
载设备上运行的应用以及各种 ECU 之间的通信链路组成\cite{ruddle2009deliverable}。如图 4.3 所示,我们根据 STRIDE 关键字修改了攻击树。值
得注意的是,这里的攻击资产目标包含两种情况:一种
是高层次、抽象的资产目标,如我们提出的网络模型的
三个层次,另一种是具体的资产目标实体,如 CAN、ECU
等。
\subsection{基于STRIDE的攻击树建模}
确定系统的资产目标后,根据 STRIDE 关键字定义
的六类威胁进行威胁识别。我们并不试图重现 STRIDE
威胁建模的过程,而是使用其关键字来指导我们构建更
全面的攻击树,因此数据流图(DFD)在这里没有使用。通过这种方式,我们可以执
行完整的攻击树建模,并且不能忽略关键的安全威胁。
\begin{figure}
    \centering
    \includegraphics[scale=0.5]{resources/img/a14.jpg}
    \caption{SATT攻击树}
  \end{figure}
\newline
在图 4.3 中,攻击树中的根节点用 G 表示,子节点可
以分为两种:攻击资产目标节点和威胁节点。它们可以
分别用 $A_i$(i = 1,2,…,n)和 $T_i$(i = 1,2,…,n)
来表示。叶子节点代表的攻击事件或方法称为原子攻
击,标记为 $E_i$(i = 1,2,…,n)。实现根节点攻击目
标的一系列原子攻击的组合被定义为攻击序列 $P_i$ (i =
1,2,… , n)。
\subsection{攻击序列的量化}
攻击树模型的结果受叶节点概率的影响,因此叶节点的
概率值直接影响威胁分析的准确性。由于攻击序列是代
表完整攻击的一系列攻击叶节点的组合,因此这里对攻
击序列进行了量化。计算攻击序列的概率。攻击的可能
性受多种因素影响。最近关于安全指标的可靠性的工作
\cite{samuel2020evaluating}表明,缺乏准确清晰的安全指标定义会导致指
标的主观性和偏差。攻击序列概率的安全度量不能任意
确定。因此,我们使用HEAVENS中定义的安全指标,这是相
对客观的。在这里,安全指标也称为安全属性。这个属
性在决策域,不在安全域。我们为每个攻击序列分配四
个安全属性:专业技能、TOE 知识、机会窗口和设备。采
用多属性效用理论将前面讨论的属性转化为效用值,以
实现攻击目标。这计算攻击序列概率的等式如(1)所示。
\begin{equation}
    P_i=W_x \times U_{x_i}+W_k \times U_{k_i}+W_w \times U_{w_i}+W_e \times U_{e_i}
    \end{equation}
\newline
这里解释下多属性效用理论\cite{winterfeldt1975multi}: 多属性决策又称为有限多目标多指标决策,是一种综合多属性的多属性决策问题,并对其进行优选(效率)或对其进行分类。其理论与方法已被广泛地运用于工程、技术、经济、管理、军事等各个方面。
\newline
关于等式的定义:
其中 $i$ 表示任何攻击序列。$P_i$代表攻击序列出现的概
率。$X_i$ 是专家。$K_i$ 代表关于TOE的知识。$W_i$ 是攻击窗口,$E_i$ 代表装备。$W_f$ 代表攻击难度的权重。 $W_x$是专业
知识的权重,$W_k$是关于 TOE 的知识的权重,$W_w$ 是攻击窗口的权重,$W_e$ 是设备的权重。这四个权重之和为
1。$U_xi$ 代表提出了专业知识的效用价值。$U_{k_i}$ 是公用事业关于 TOE 的知识值,$U_{w_i}$代表机会窗口的效用值和$U_{e_i}$ 代表展示设备的实用价值。
\newline
值得关注权重向量$W$是针对每个攻击序列的而不是整个系统,
因为攻击行为不同的攻击序列所代表的效果也不同论安全属性权重。
我们不能把重量向量作为TOE的常数。
此外,我们可以分析 $X_i$、$K_i$ 、$W_i$ 和$E_i$ 与 $U_xi$ 、$U_{k_i}$、$U_{w_i}$ 和 $U_{e_i}$ 成反比。因此,对于计算的方便性,它们之间的关系取为 U(x) = 1/x。
这里,当计算攻击序列的概率时,涉及到四个安全
属性,因此这是必要的制定相应的评分标准对其进行评估。采用的评分标准如表 4.1 所示。
\begin{table}
  \caption{评分标准表}
\begin{center}
  \begin{tabular}{|l|l|l|l|}
    \hline 专业知识 &价值&关于TOE&价值\\
    \hline 外行 & 1  &公众 & 1 \\
    \hline 精通 & 2  &受限 & 2 \\
    \hline 专家 & 3 & 敏感 & 3 \\
    \hline 多个专家 & 4  & 临界的  & 4 \\
    \hline 攻击窗口&价值&设备&价值 \\
    \hline 临界 & 1  &标准 & 1 \\
    \hline 高 & 2  &专门 & 2 \\
    \hline 中  & 3  & 定制的  & 3 \\
    \hline 低  & 4  &多个定制 & 4 \\
    \hline
  \end{tabular}
\end{center}
\end{table}
\subsection{ 基于 FAHP的安全属性权重的确定}
对于不同的攻击序列,专家意见的权重、关于 TOE 的知
识、机会窗口和设备。FAHP\cite{liu2020review} \cite{kubler2016state} 是对定性问题进行定量分析
的一种简单而直观的方法。表 3 所示的 0.1-0.9 标度标
准用于定量描述每个属性的相对重要性。
在 FAHP 中,应该基于特定的标度准则,通过两两比
较元素来构造模糊判断矩阵。如果模糊判断矩阵不一
致,则应将其转换为模糊一致判断矩阵。最后,利用模
糊一致判断矩阵计算各元素相对重要性的权重。
\begin{table}
  \caption{权重判断表}
\begin{center}
    \begin{tabular}{|p{0.15\textwidth}<{\raggedright}|p{0.3\textwidth}<{\raggedright}|p{0.3\textwidth}<{\raggedright}|}
     \hline
      比例 & 定义 & 描述 \\\hline
      0.5 & 同等重要 & 这两个因素同等重要 \\\hline
      0.6 & 稍微重要 & 其中一个元素比另一个稍微重要一些 \\\hline
      0.7 & 明显重要 & 其中一个元素显然比另一个更重要 \\\hline
      0.8 & 更重要的 & 一个元素比另一个更重要 \\\hline
      0.9 & 极重要的 & 一个元素比另一个元素极其重要 \\\hline
      0.1,0.2,0.3,0.4 & 相反比较 & 如果将元素$a_i$与元素$a_j$进行比较,则得到判断$r_{i j}$。那么,如果$a_j$与$\alpha_i$比较,则判断为$r_{j i}=1-r_{i j}$ \\\hline  
    \end{tabular}
  \end{center}
\end{table}
根据表 3 可以得到模糊判断矩阵 R
$$
R=\left[\begin{array}{cccc}
r_{11} & r_{12} & \cdots & r_{1 n} \\
r_{21} & r_{22} & \cdots & r_{2 n} \\
\cdots & \cdots & \cdots & \cdots \\
r_{n 1} & r_{n 2} & \cdots & r_{n n}
\end{array}\right]
$$
模糊判断矩阵的一致性应按以下性质进行检验
\begin{equation}
    r_{i i}=0.5, \quad i=1,2, \ldots, n
    \end{equation}
    
    \begin{equation}
    r_{i j}=1-r_{j i}, \quad i, j=1,2, \ldots, n
    \end{equation}
    
    \begin{equation}
    r_{i j}=r_{i k}-r_{j k}+0.5, \quad i, j, k=1,2, \ldots, n
    \end{equation}
    
如果矩阵 R 满足三个条件,则该矩阵是模糊一致判
断矩阵。如果不是,为了确保两个元素的相对重要性的
一致性,使用基于等式 4 的算术平均来调整矩阵的各个
元素。公式如下:
\begin{equation}
    r_{i j}^{\prime}=\frac{1}{n} \sum_{k=1}^n\left(r_{i k}-r_{j k}+0.5\right)
    \end{equation}

r 是调整后的模糊一致判断矩阵 Ru 的元素。然后,n
是矩阵的阶。每个安全属性的权重向量 $W_i$  可以通过最
小二乘法对矩阵 Ru 进行归一化来计算。方程式如下:
\begin{equation}
    W_i=\frac{1}{n}-\frac{1}{2 a}+\frac{1}{n a} \sum_{k=1}^n r_{i k}^{\prime}
    \end{equation}

    其中 a 是重量差异的影响因子,以及 此处需补充: 
    \begin{equation}
        W_i=\frac{2}{n(n-1)} \sum_{k=1}^n r_{i k}^{\prime}-\frac{1}{n(n-1)}
        \end{equation}
        
        根据先前的方法,最后我们可以计算每个攻击序列的安全属
性权重向量 $W$。
\begin{equation}
    W=\left(W_1, W_2, W_3, \ldots, W_n\right)^T
    \end{equation}

\subsection{计算攻击序列的概率并进行风
险评估}

攻击序列代表一组攻击行为。它出现的概率表示在每个
攻击场景中攻击目标的可能性。当攻击序列所代表的攻
击行为实现时,一个攻击事件就完成了。我们可以根据
等式 4.1 计算攻击序列的概率,进行 TARA 来发现安全关键
系统中的威胁和漏洞,然后部署相应的安全防御机制。
\newline


\section{应用场景: 分析车载网络系统的安全威胁}
本节通过实例应用SATT安全威胁分析方法。
\subsection{车载网络的安全威胁分析}
\begin{figure}
  \centering
  \includegraphics[scale=0.5]{resources/img/i31.jpg}
  \caption{实例节点定义图}
\end{figure}
基于提出的三层车载网络模型,可以先确定三个高级的资产对象类别,然后根据 EVITA\cite{dominic2016risk}
的资产定义导出具体的攻击资产。在 STRIDE 关键字的帮
助下识别安全威胁全面攻击树。由于车载网络的整个攻击树很难追踪,我
们以网络通信层的 CAN/CAN FD\cite{reindl2021comparative} \cite{zago2017quantitative} 攻击为例来演示所提出
的方法。图4.3显示了基于物理访问攻击的常见场景,以说明所提出
的方法的有效性。
表 4.4显示了攻击树中每个节点的含
义。
\begin{table}
  \caption{攻击树节点定义表}
\begin{center}
    \begin{tabular}{|l|l|l|l|}
      \hline 节点 & 定义 & 节点 & 定义 \\
      \hline $G$ & 攻击车载网络 & $E_2$ & 插入虚假数据 \\
      \hline $A_1$ & 攻击网络通信 & $E_3$ & 非法物理访问 \\
      \hline $A_2$ & 攻击 CAN/CAN FD & $E_4$ &数据修改 \\
      \hline $T_1$ & 欺骗攻击 & $E_5$ & 非法物理访问 \\
      \hline $T_2$ & 篡改攻击 & $E_6$ & 重放数据 \\
      \hline $T_3$ & 重放攻击 & $E_7$ & 非法物理访问 \\
      \hline $T_4$ & 嗅探攻击 & $E_8$ & 监听和拦截数据 \\
      \hline $T_5$ & 拒绝服务 & $E_9$ & 非法物理访问 \\
      \hline $E_1$ & 非法物理访问 & $E_{10}$ & 连续发送高优先级数据包 \\\hline
      \end{tabular}
  \end{center}
\end{table}
我们使用表 3 中的标度标准对每个攻击序列的安全
属性进行评分。为了证明该方法的有效性和避免个人评
价的主观性,我们进行了问卷调查,主要面向了TARA相关领域的专家学者等。参与者的TARA经
验越多。结果就越准确可靠。最后,取其结果的平均值
并取整。评分结果如表 4.4 所示。
\begin{table}
  \caption{攻击序列评分表}
\begin{center}
  \begin{tabular}{|l|l|l|l|l|}
    \hline 攻击序列 &专家意见&TOE&最佳时机&装备资产\\
    \hline P1\{E1, E2\} & 2  &4 & 2 & 2 \\
    \hline P2\{E3, E4\} & 2  &2 & 4 & 2 \\
    \hline P3\{E5, E6\} & - & 2 & 4 & 2 \\
    \hline P4\{E7, E8\}  & -  & 2  & 4 & 2 \\
    \hline P5\{E9, E10\}  & 2  & 2  & 4 & 2 \\
    \hline
  \end{tabular}
\end{center}
\end{table}
图 4.3 显示了五个攻击序列可以攻击 CAN/ CAN FD。然
后,用 FAHP 来判断和比较每个攻击序列的安全属性权
重,得到它们的模糊判断矩阵如下:
$$
\begin{gathered}
R_{P_1}=\left[\begin{array}{llll}
0.5 & 0.6 & 0.4 & 0.6 \\
0.4 & 0.5 & 0.3 & 0.5 \\
0.6 & 0.7 & 0.5 & 0.7 \\
0.4 & 0.5 & 0.3 & 0.5
\end{array}\right], \quad R_{P_2}=\left[\begin{array}{llll}
0.5 & 0.6 & 0.4 & 0.6 \\
0.4 & 0.5 & 0.3 & 0.5 \\
0.6 & 0.7 & 0.5 & 0.7 \\
0.4 & 0.5 & 0.3 & 0.5
\end{array}\right] \\
R_{P_3}=\left[\begin{array}{llll}
0.5 & 0.5 & 0.3 & 0.5 \\
0.5 & 0.5 & 0.3 & 0.5 \\
0.7 & 0.7 & 0.5 & 0.7 \\
0.5 & 0.5 & 0.3 & 0.5
\end{array}\right], \quad R_{P_4}=\left[\begin{array}{llll}
0.5 & 0.4 & 0.2 & 0.4 \\
0.6 & 0.5 & 0.3 & 0.5 \\
0.8 & 0.7 & 0.5 & 0.7 \\
0.6 & 0.5 & 0.3 & 0.5
\end{array}\right] \\
R_{P_5}=\left[\begin{array}{llll}
0.5 & 0.5 & 0.3 & 0.5 \\
0.5 & 0.5 & 0.3 & 0.5 \\
0.7 & 0.7 & 0.5 & 0.7 \\
0.5 & 0.5 & 0.3 & 0.5
\end{array}\right]
\end{gathered}
$$

上述模糊判断矩阵都满足模糊一致性判断矩阵的条
件,因此不需要通过等式 4.5 进行一致性转换。

然后我们直接用等式 4.7 计算所有攻击序列的安全属性权
重向量。结果如公式 4.9 所示。
\begin{equation}
    W_{P_i}=\left[\begin{array}{lllll}
    0.267 & 0.267 & 0.217 & 0.167 & 0.217 \\
    0.200 & 0.200 & 0.217 & 0.233 & 0.217 \\
    0.333 & 0.333 & 0.350 & 0.367 & 0.350 \\
    0.200 & 0.200 & 0.216 & 0.233 & 0.216
    \end{array}\right]
    \end{equation}

将等式 9 和表 5 中的值代入等式 1,可以获得每个攻
击序列的概率。结果如表 6 所示。
\begin{table}
  \caption{基于FAHP的攻击序列可能性表}
\begin{center}
    \begin{tabular}{|l|l}
      \hline 攻击序列 & 可能性 \\
      \hline P1 & 0.372 \\
      \hline P2 & 0.372 \\
      \hline P3 & 0.413 \\
      \hline P4 & 0.492 \\
      \hline P5 & 0.413 \\\hline
      \end{tabular}
  \end{center}
\end{table}
从表 6 中的数据可以看出,攻击序
列 P4 出现的概率更高,这意味着嗅探攻击比其他攻击
更简单、更容易执行。因此,在 CAN/CAN FD 中考虑和
部署安全机制时,我们应该首先关注这个问题。
\subsection{有效性比较}
我们通过FAHP和传统的AHP进行比较,证明使用FAHP的优点和和有效性。
在AHP中,用判断矩阵R来表达问题中各个因素或方案之间的相对重要性。矩阵R满足判断矩阵需要满足的数学性质是一致性、对称性和正定性。其中,一致性是AHP方法的核心,它要求判断矩阵中的每个元素都必须符合矩阵的逻辑关系,否则就存在矛盾和不一致性。即以下数学表达式:
\begin{equation}
  r_{i i}=1, \quad i=1,2, \ldots, n \\
\end{equation}

\begin{equation}
  r_{i j}=\frac{1}{r_{j i}}, \quad i, j=1,2, \ldots, n
\end{equation}

\begin{table}
  \caption{重要性比例标度标准}
\begin{center}
    \begin{tabular}{|p{0.15\textwidth}<{\raggedright}|p{0.3\textwidth}<{\raggedright}|p{0.3\textwidth}<{\raggedright}|}
     \hline
      比例 & 定义 & 描述 \\\hline
      1 & 同等重要 & 这两个因素同等重要 \\\hline
      3 & 稍微重要 & 其中一个元素比另一个稍微重要一些 \\\hline
      5 & 明显重要 & 其中一个元素显然比另一个更重要 \\\hline
      7 & 更重要的 & 一个元素比另一个更重要 \\\hline
      9 & 极重要的 & 一个元素比另一个元素极其重要 \\\hline
      2,4,6,8 & 相反比较 & 如果将元素$a_i$与元素$a_j$进行比较,则得到判断$r_{i j}$。那么,如果$a_j$与$\alpha_i$比较,则判断为$$r_{j i}=1 / r_{i j}$$
      \\\hline  
    \end{tabular}
  \end{center}
\end{table}
与 FAHP 不同,表4.6中所示的 1-9 标度标准用于比较攻击序列的四个安全属性的权重。
根据标度准则,我们把上节中所用到的五个攻击序列比较了每个攻击序列的安全属性权重。判断矩阵如下获得:
$$
\begin{aligned}
& R_{P_1}^{\prime}=\left[\begin{array}{cccc}
1 & 3 & 1 / 3 & 3 \\
1 / 3 & 1 & 1 / 5 & 1 \\
3 & 5 & 1 & 5 \\
1 / 3 & 1 & 1 / 5 & 1
\end{array}\right], \quad R_{P_2}^{\prime}=\left[\begin{array}{cccc}
1 & 3 & 1 / 3 & 3 \\
1 / 3 & 1 & 1 / 5 & 1 \\
3 & 5 & 1 & 5 \\
1 / 3 & 1 & 1 / 5 & 1
\end{array}\right] \\
& R_{P_3}^{\prime}=\left[\begin{array}{cccc}
1 & 1 & 1 / 5 & 1 \\
1 & 1 & 1 / 5 & 1 \\
5 & 5 & 1 & 5 \\
1 & 1 & 1 / 5 & 1
\end{array}\right], \quad R_{P_4}^{\prime}=\left[\begin{array}{cccc}
1 & 1 / 3 & 1 / 7 & 1 / 3 \\
3 & 1 & 1 / 5 & 1 \\
7 & 5 & 1 & 5 \\
3 & 1 & 1 / 5 & 1
\end{array}\right] \\
& R_{P_5}^{\prime}=\left[\begin{array}{cccc}
1 & 1 & 1 / 5 & 1 \\
1 & 1 & 1 / 5 & 1 \\
5 & 5 & 1 & 5 \\
1 & 1 & 1 / 5 & 1
\end{array}\right]
\end{aligned}
$$

然后,我们需要根据等式 12 来判断上述矩阵的一致性。
\begin{equation}
\text{CR}=\frac{\text{CI}}{\text{RI}} < 0.1
\end{equation}

其中,$\text{CI}$表示一致性指标,$\text{RI}$表示随机一致性指标。

一致性指标$\text{CI}$可以通过计算特征向量的最大特征值$\lambda_{max}$和判断矩阵的大小$n$得到:
\begin{equation}
\text{CI}=\frac{\lambda_{max}-n}{n-1}
\end{equation}

当使用公式4.13时,需要先计算出矩阵的最大特征值 $\lambda_{max}$,然后使用该公式计算出置信区间 $\text{CI}$。这个置信区间的含义是,以给定置信水平(例如95\%)对最大特征值进行估计时,真实值落在此区间的概率是给定置信水平。如果置信区间越小,就意味着估计的精度越高。
此外,随机一致性指标$\text{RI}$是通过参考随机一致性指标表格中如表4.7所示的数值得到的。如果一致性比例$\text{CR}$小于0.1,则认为判断矩阵具有可接受的一致性,否则需要重新构建判断矩阵。

\begin{table}
  \centering
  \caption{随机一致性指标表}\label{table1}
  \begin{tabular}{|l|l|l|l|l|l|l|l|l|l|}
  \hline
  $n$ & 1 & 2 & 3 & 4 & 5 & 6 & 7 & 8 & 9 \\
  \hline 
  $\mathrm{RI}$ & 0 & 0 & 0.58 & 0.9 & 1.12 & 1.24 & 1.32 & 1.41 & 1.45 \\
  \hline
  \end{tabular}
  \end{table}
  
  经检验,上述判断矩阵均满足一致性条件。因此可以计算出矩阵的归一化特征向量,这就是我们需要的权向量。结果如等式4.14所示。
  \begin{equation}
    W_{P_i}^{\prime}=\left[\begin{array}{lllll}
    0.250 & 0.250 & 0.128 & 0.062 & 0.128 \\
    0.095 & 0.095 & 0.128 & 0.151 & 0.128 \\
    0.560 & 0.560 & 0.639 & 0.636 & 0.639 \\
    0.095 & 0.095 & 0.105 & 0.151 & 0.105
    \end{array}\right]
    \end{equation}
    

    \begin{equation}
      \begin{array}{l|l}
  \text { 攻击序列 } & \text { 可能性 } \\
  \hline P_1 & 0.318 \\
  P_2 & 0.318 \\
  P_3 & 0.340 \\
  P_4 & 0.372 \\
  P_5 & 0.340 \\
  \hline
  \end{array}
\end{equation}

\begin{figure}
  \centering
  \includegraphics[scale=0.5]{resources/img/fahp.jpg}
  \caption{FAHP和AHP}
\end{figure}
  
  攻击序列的安全属性得分与FAHP一致。因此,每个攻击序列的概率也可以通过将等式4.14代入等式1并结合表5来获得。结果如表4.15所示。
  两种方法的比较结果如图4.5所示。可以看出,两种方法评价结果的趋势大致相同。P1和P2有相同的概率。在这两种情况下,P4的可能性最大。成功攻击可能性较高的攻击序列所代表的攻击行为能够被反映出来,说明了两种方法的有效性。但由于两者的权重向量不同,具体概率值也不同。构造判断矩阵时,考虑各属性的重要度梯度较小,权重值应与重要度一致。显然,FAHP的梯度小于层次分析法的梯度。在权重方面,FAHP比层次分析法更客观,但层次分析法的缺点,如判断矩阵的一致性难以检查,在这里没有太大影响。在我们的方法中,两者都可以用来计算攻击事件发生的概率。
  
\subsection{SATT威胁建模与其他威胁建模对比评估}
首先,我们考虑STRIDE模型和我们提出的SATT威胁建模方法的优劣。STRIDE模型易于理解和教授,因此在非安全和非技术团队成员中应用广泛。它能够快速识别可能影响正在建模的系统的高级威胁,并且具有较快的执行速度。相比之下,SATT模型不仅包括STRIDE模型,还通过攻击树模型进行威胁分析。SATT模型的场景涵盖了很多安全场景,因此更适用于对复杂场景进行验证,化繁为简。特别是在智能网联汽车场景下,SATT模型表现更出色。

其次,与常用的基于AHP的攻击树模式相比较,SATT模型首先将STRIDE的元素添加到攻击树中,通过对资产威胁进行分类,减少了构建攻击树的盲目性。此外,该方法为攻击事件分配多个属性,并定量描述每个属性的重要性,然后使用数学方法计算权重,最终得到叶节点所代表的攻击事件的概率。相比基于攻击场景或历史数据的方法,该方法具有一定的灵活性和优势。因为历史数据有时难以获得,可能需要长时间的大量实验或统计,并且还需要对复杂数据进行分析处理。我们的FAHP方法基于专家评分和严格的数学计算,使计算攻击事件的概率变得更加容易。此外,我们考虑各种因素,包括攻击者的能力,而不仅仅是攻击场景。但是,与基于数据的方法相比,该方法的主观性不可避免,因为它涉及各种因素的相对权重和攻击事件的得分,这些大多依赖于专家的主观判断。
\section{本章小结}
在这一章中,重点对智能网联汽车领域的威胁建模分析进行了阐述,以传统的STRIDE模型和攻击树模型为基础,提出了一种被称为SATT全新的威胁建模方法。
对其建模原理和步骤进行了详细的说明,并以一辆真实的智能网联汽车汽车为例进行威胁建模;
然后还对实例威胁车载网络进行了威胁建模分析,对威胁进行了更深一步的判断;最后对比分析了传统STRIDE威胁建模以及SATT威胁建
模方法优缺点。