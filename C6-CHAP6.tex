\chapter{总结与展望}
\label{ch6}

\section{工作总结}
本文从智能网联汽车安全架构入手,首先展示现有 ICV 领域使用的通信技术、
基础设施以及网络拓扑;然后通过大量的调查对 ICV 面临的安全威胁进行系统分类
和展示,给出多种典型攻击的案例;紧接着分别使用传统威胁建模方法和自己提出
的创新威胁建模方法对 ICV 进行威胁建模分析,并且对两者进行了对比评估,并且
还对无人研究的车载 TSP 系统进行威胁建模和安全分析;最后通过漏洞挖掘和利用,
编写渗透脚本,成功对一款新型 ICV 实施了远程入侵,证明当前 ICV 确实存在诸多
的安全问题,并给出了几点针对性的预防措施。最终,设计了一个关于 ICV 的远程
入侵与安全防御系统,将其应用在我们的实验汽车中,达到了较为理想的效果。
本文也存在一些不足之处,一是只是从 TSP 系统的远程攻击的角度去进行渗透
攻击,限于篇幅,没有对蓝牙等近距离通信展开实质性研究,日后需要加强 ICV 其
他领域的窥探;二是新提出的威胁建模方法,只是成功的应用在了 ICV 领域,至于
其他场景的此新型威胁建模方法,有待验证和改进。

\section{未来展望}

威胁建模的结果显示目标 ICV 并不安全,其 TSP 系统也存在很多安全隐患,实
验用车也存在很多漏洞,这都间接导致车辆被非法利用和远程控制;我们的渗透攻
击实验也成功说了此问题。随着时代发展,ICV 必然大放光彩,在未来社会必将大
显身手。但是其安全问题也必将凸显,增强 ICV 的安全性也不是一蹴而就的,需要
我们齐心协力,共同并进