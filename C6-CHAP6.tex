\chapter{总结与展望}
\label{ch6}

\section{工作总结}
本文从智能网联汽车安全架构入手,首先展示现有 ICV 领域使用的通信技术、
基础设施以及网络拓扑;然后通过大量的调查对 ICV 面临的安全威胁进行系统分类
和展示,给出多种典型攻击的案例;紧接着分别使用传统威胁建模方法和自己提出
的创新SATT威胁建模方法对 ICV 进行威胁建模分析,并且对传统的STRIDE和攻击树模型进行了对比评估,并且
还对无人研究的车载TSP系统进行威胁建模和安全分析;最后通过漏洞挖掘和利用,
编写渗透脚本,成功对一款新型 ICV 实施了远程入侵,证明当前 ICV 确实存在诸多
的安全问题。
本文也存在一些不足之处,一是只是从 TSP 系统的远程攻击的角度去进行渗透
攻击,限于篇幅,没有对其他方向的攻击和威胁进行深入研究和探讨,日后需要加强 ICV 其
他领域的窥探;二是新提出的威胁建模方法,只是成功的应用在了 ICV 领域,至于
其他场景的此新型威胁建模方法,有待验证和改进。

\section{未来展望}

威胁建模的结果显示目标 ICV 并不安全,其 TSP 系统也存在很多安全隐患,实
验用车也存在很多漏洞,这都间接导致车辆被非法利用和远程控制;我们的渗透攻
击实验也成功说了此问题。虽然本文进行的是车载网络的安全威胁分析,但是
这种方法并不局限于汽车领域。它也可以应用于其他信
息物理系统。未来,我们将优化方法,进行全面的安全
威胁分析,有助于在汽车网络安全的发展中更早地发现
威胁和漏洞。