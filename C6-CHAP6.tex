\chapter{总结与展望}
\label{ch6}

\section{工作总结}
论文从智能网联汽车的安全架构开始,首先展示现有 ICV 领域使用的通信技术、
基础设施以及网络拓扑;然后通过大量的调查对 ICV 面临的安全威胁进行系统分类
和展示,列出多种典型的攻击和威胁面;并用传统STRIDE给智能网联汽车的IVI系统进行了威胁建模,第四章提出了自己的威胁建模方法并进行了实例验证,最后给出了和STRIDE模型的对比评估;最后通过漏洞挖掘和利用,
编写渗透脚本,成功对一款新型 ICV 实施了远程入侵,证明当前 ICV 确实存在诸多的安全问题。
本文也存在一些不足之处,一是只是从 TSP 系统的远程攻击的角度去进行渗透
攻击,限于篇幅,没有对其他方向的攻击和威胁进行深入研究和探讨,日后需要加强 ICV 其
他领域的窥探;二是新提出的威胁建模方法,只是成功的应用在了 ICV 领域,至于
其他场景的此新型威胁建模方法,有待验证和改进。

本文在对ICV进行了远程攻击的基础上,对当前ICV存在的许多安全问题进行了分析探讨。
文章也有不足之处,一是只对IVI系统的远距离攻击做了较深的探讨;
由于篇幅所限,并没有对其他方面的攻击和威胁进行详细的分析和探讨,今后还需要对车联网、物理等方向的远程攻击等进行进一步的研究。
第二,我们提出的安全威胁建模有效性尚待验证与改进,并且希望可推广到更多安全威胁建模领域。

\section{未来展望}
通过对威胁模型的分析,表明了智能网联汽车的安全性目前还不算高, 本文提出的问题都会间接或直接地造成车辆被非法使用和远程操控;
虽然本文进行的是车载网络的安全威胁分析,但是
这种方法并不局限于汽车领域。它也可以应用于其他信
息物理系统。未来,我们将优化方法,进行全面的安全
威胁分析,有助于在汽车网络安全的发展中更早地发现
威胁和漏洞。