\chapter{总结与展望}
\label{ch6}

\section{工作总结}
智能网联汽车已经成为汽车行业的一个重要发展方向,具有不可忽视的安全风险。安全威胁建模是一种识别和评估系统威胁的方法,通过建立威胁模型来帮助确定安全性问题并提出解决方案。在智能网联汽车的设计和开发中,进行安全威胁建模非常重要。

本文从智能网联汽车的安全架构入手,将现有ICV领域中遇到的安全风险和主要攻击手段进行归纳,并对STRIDE安全威胁建模流程进行了介绍。首先,我们对智能网联汽车的IVI娱乐信息系统的基本功能架构进行了梳理,对其系统的安全威胁进行了深入全面的分析。最后,利用STRIDE安全威胁模型对其进行建模,并对其中三条数据流进行威胁评级。

本文提出了名为SATT安全威胁建模方法,通过结合攻击树模型和STRIDE模型,再通过FAHP模糊层次分析法给出攻击序列的发生概率,更利于风险评估人员指定缓解方案。此外,我们还比较了FAHP和传统的AHP的有效性,发现FAHP更为客观。

最后,我们随后进行了真实的物理实验,通过三个攻击路径:蓝牙信号篡改、破解车辆互联App、抓包车辆IVI系统流量对车辆系统进行了远距离攻击,并给出详细的背景知识和实施流程。结果均取得不错的实验结果。结果表明安全威胁绝不是纸上谈兵,目前智能网联汽车存在不可忽视的安全威胁问题。

\section{未来展望}
通过对威胁模型和实际的智能网联真实汽车的攻击,表明了智能网联汽车的安全性还存在很大问题, 本文提出的问题都会间接或直接地造成车辆被非法使用和远程操控;

本文的缺陷在于,模型的安全标准评分过于依赖主观和专家评分,模型有效性还有待提高和进一步验证。

未来智能网联汽车的发展趋势将带来更为复杂的安全威胁,需要完善安全威胁建模方法,加强对异常行为的识别和响应能力,提高建模方法的普适性和可操作性。目前的不足包括缺乏对新兴攻击手段的考虑、建模需要大量的专业知识和经验、难以普及推广等问题。因此,应进一步推进智能网联汽车安全技术的发展和普及。

未来攻击者可能会利用机器学习和人工智能技术来对智能网联汽车进行攻击,从而产生更加复杂和难以预测的威胁。因此,未来的安全威胁建模需要考虑这些新型攻击方式,并通过人工智能技术提高自身的防御能力。