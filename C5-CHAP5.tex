\chapter{TSP 渗透测试与远程入侵}
\label{ch5}
\section{攻击模型}
实验环境包括某知名品牌汽车一台,装有远程控制 App 的手机一部,无线路由
器一个以及一台装有无线网卡的电脑。软件环境:手机环境为安卓5.1.1,App 版本
1.1.8(发布日期:2017.8.21)。由于前期通过购买传感器、ECU、单片机和 IVI 等设
备,手动搭建了智能网联汽车的安全攻防演练平台,因而对车内网络和环境有了更
加深刻的理解,并且对各个部分的通信机制和原理有了更加全面的认识。这样,后
期购买的实际的汽车,在不用拆车的情况下,就可以掌握其内部工作机理。购买的
汽车如图 5.1 所示,是 2017 款的 SUV,已经具有网联远控的功能,在第四章对其进
行威胁建模的基础上,对其进行实际攻击测试,证明当前的 ICV 领域确实还存在很
多不容忽视的安全问题。、
\newline
整体的流程如图 5.2 所示。大致分为环境搭建、数据分析、漏洞利用和逆向工程
几个步骤,与后文的章节基本上是相互对应的。实验用车经过多次测试,已经确定
存在相关漏洞;首先需要分析 TSP 与 App 之间的交互数据,主要是为了确定普通用
户登陆过程存在的问题已经登陆的格式,并且在漏洞利用部分,编写相应的代码进
行测试,判断能否找到账号,其中还要考虑不要触发服务器的入侵检测系统;在逆
向工程部分,对 App 进行初级反编译,找出可能存在的加固方式,然后使用动态方
式 Dump 内存中的 DEX 文件,再进行反汇编和动态调试,从汇编代码中找到指令以
及服务器的认证格式;最后用相应的代码进行验证。
\newline

\section{流量分析和漏洞利用}
通过无线信道抓取流量进行分析,我们有两种方法搭载热点:a.使用一个普通的
无线路由器作为无线热点;b.使用无线网卡连接笔记本电脑,配置一个无线热点。对
于这两种搭建无线局域网的方式,我们使用不同的方法进行抓包分析。

\subsection[]{数据包抓取与分析
}

\section{title 1}
\section{title 2}
\section{title 3}
\section{本章小结}