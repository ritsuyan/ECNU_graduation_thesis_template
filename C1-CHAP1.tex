\chapter{绪\hskip 0.4cm 论}
\label{ch1}
\section{研究背景及意义}
21世纪以来,随着移动互联网、物联网、工业智能化等技术的快速发展,汽车产
业工业也迎来了大变革,汽车由传统的交通工具向出现、娱乐、智能等方向发展。汽车的智能和互联性正在推动车载网络系统的复
杂性和多样性的增加。智能网联汽车目前正处于渗透率快速提升的阶段,智能网联汽车能够将手机的百万量级的应用融合到汽车中,
实现汽车应用的量级突破。
智能网联汽车有望继智能手机之后,成为新一代的超级终端。
根据中商情报网数据显示\cite{zhao},2016-2020年我国智能网联汽车产业规模呈现连续上涨趋势,2020年产业规模增长到了2556亿元,同比增长54.3\%。
预计,2022年智能网联汽车产业规模将超3500亿元。

汽车的智能和互联性正在推动车载网络系统的复杂性和多样性的增加。信息通信技术给传统车辆带来了巨大的创新,也给现代车辆带来了重大的安全挑战。原本隔离的车载网络连
接到外部,增加了智能网联汽车(Intelligent Connected Vehicle,ICV)的攻击面,并带来了新的安全风险。汽车网络安全引起了公众的关注。
Checkoway 等人\cite{checkoway2011comprehensive}从网络安全的角度描述了现代车辆
的安全威胁模型和外部攻击媒介,并研究了如何进入车辆
的内部网络。2015 年,查理·米勒和克里斯·瓦拉塞克
(Chris Valasek)演示了如何通过攻击IVI(In-Vehicle Infotainment, 车载娱乐信息系统)来入侵一辆吉普车的车载网络,并导致车辆的异常行
为,导致 140 万辆汽车被召回。与此同时,自动网络安全
开始引起公众的广泛关注\cite{miller2015remote}。Koscher 等人 \cite{koscher2010experimental} 通过嗅探
并向CAN(ControllerArea Network, 控制器局域网络)总线注入恶意代码,通过Dos
(Denial of Service, 拒绝服务)攻击阻断电子控制单元(ECU)通信,从而控制车辆的
多个功能模块。在\cite{liu2017vehicle}中,针对 CAN 的安全缺陷,提出了广
播、无认证、无加密、仲裁优先、接口易访问等五种对
CAN 的攻击方法。除了通过 CAN 攻击车辆,本地互联网络
(LIN) \cite{deng2017security},FlexRay \cite{takahashi2017automotive} \cite{gu2016security} 总线系统和其他外部接
口,例如通用串行总线(USB),蓝牙,Wi-Fi,蜂窝和车载OBD(On-Board Diagnostics,车载诊断系统),也可用于非法访问车载网络\cite{mousa2016lightweight}。
安全威胁远不止前面提到的那些。因此,有必要在
车辆的早期开发阶段进行安全威胁分析,以找出威胁漏
洞,实施安全措施,预防安全问题。

因此,以智能网联信息系统为载体的智能网联汽车,其威胁建模的研究意义将是十分重要和深远的, 主要体现在以下几点:
\begin{itemize}
  \item 智能网联汽车的信息安全对车载驾驶人身安全和财产安全至关重要,因此,对智能网联汽车的攻击与威胁问题进行深入的研究,可以促进整车厂商对其进行改进,以保障人民的生命和财产的安全。
  \item 通过及时发现智能网联汽车网络系统的缺陷,可以将智能网联汽车所面临的安全风险进行全面、明确的展现,为安全开发人员提供必要的信息,并能及时发现危险来源,为更安全的保护工作做好准备。  。
  \item 针对智能网联汽车提出的新的威胁建模方法, 并成功地实现了一款商用SUV的远程入侵实验,证明了在智能网联汽车中仍有很多安全隐患,弥补了车联网研究领域的短板。  
\end{itemize}
\section{国内外研究现状}
关于智能网联汽车威胁建模及安全风险评估的问题,其国内外的相关
研究现状如下:

\subsection{智能网联汽车威胁的研究}
智能网联汽车由于具有高度的集成化和智能化,使其具有更多的功能接口和更复杂的系统复杂性。
随着时间的推移和黑客的研究深入,暴露在外界的威胁也会越来越多。从有线、无线、到汽车
各种接口,使汽车与外部世界的接触方式日益增多,同时也必然会出现各方面的网络安全问题、
各种漏洞,例如软件漏洞、代码漏洞、硬件接口漏洞等。
而且,它的攻击方式也会越来越复杂,想要破解它的攻击方式也会越来越困难。
目前国内外学者对一些较为典型的漏洞使用进行了研究如下。

Checkoway\cite{checkoway2011comprehensive} 等人从网络安全的角度描述了现代车辆
的安全威胁模型和外部攻击媒介,并研究了如何进入车辆
的内部网络。他们试图通过系统分析现代汽车的外部攻击面来解
决这个问题。通过广泛的攻击媒介(包括
机械工具、CD 播放器、蓝牙和蜂窝无线电)进行远
程利用是可行的,此外,无线通信信道允许远程车
辆控制、位置跟踪、车内音频泄漏和盗窃。
CAN协议是一种广泛应用于车载网络的总线协议。Liu\cite{liu2017vehicle}等人总结了以下有效攻击方法:帧嗅探,帧伪造,
帧注入,重放攻击,拒绝服务攻击等。此外新型的智能网联汽车具有手机APP等能操控智能网联汽车本身的外部操控智能设备
因此从手机端APP等进行攻击也是目前比较流行的一个新颖攻击方法。

\subsection{智能网联汽车威胁建模的研究}
微软公司提出了STRIDE威胁模型 \cite{kohnfelder1999threats} 用于识别计算机安全威胁。用于帮助推理和发现对系统的威胁。它与可以并行构建的目标系统模型结合使用。这包括对流程、数据存储、数据流和信任边界的全面分解。

Mats Olsson等人提出了HEAVENS威胁安全模型\cite{lautenbach2021proposing},针对车辆电子电气(E/E)系统的信息安全威胁分析和风险评估的方法、流程及工具支持,可结构化和系统化的方法发现潜在威胁。
Mohsin等人\cite{mohsin2017iotriskanalyzer} 介绍了一种基于概率模型检测的形式化风险
分析框架。他们的框架能够生成系统威胁模型,用于正式计算攻击的可能性和成本。

Agadakos\cite{agadakos2017jumping}等人介绍了一种使用 Alloy 为物联网中的网络物理攻击路径建
模的方法。他们的方法最终也会产生潜在的威胁。还提出
了支持物联网威胁分析方面的非形式化方法。例如
Geneiatakis 等人。

UcedaVelez等人提出了\cite{ucedavelez2015risk} 攻击模拟和威胁分析 (PASTA) 威胁建模方法的过程。它介绍了各种类型的应用程序威胁建模,并介绍了一种以风险为中心的方法,旨在应用与定义的威胁模型、漏洞、弱点和攻击模式可能产生的影响相称的安全对策。 

从现有的研究来看,智能网联汽车的发展还不够成熟,安全隐患依然很大。
担心。特别是针对无线网络的攻击,它具有很高的隐蔽性和覆盖面,常常会给使用者造成很大的伤害。
存在着重大的安全风险和灾难后果。目前,人们对智能网联车辆的安全意识逐渐提高,
然而,目前还没有一个完整的理论体系和规范标准。
对智能联网汽车的研究信息安全关系到人身生命安全和财产安全。

\section{本文工作和主要贡献}
\paragraph{本文贡献点主要有如下三个:}

\begin{enumerate}
    \item 智能网联汽车架构及威胁分析 
    分析智能网联汽车的系统架构及其通信机
    制,并从CAN、IVI、车内网络通信几个方
    面分析智能网联汽车所面临的安全威胁。
    \item 分析了TARA中主流的威胁建模方法如HEAVENS, EVITAS, STRIDE, 攻击树等。
    研究智能网联汽车的威胁模型,结合STRIDE和攻击树模型提出一种创新的名为SATT的威胁建
模方法,对实际商用的汽车的网络架构进行了威胁建模,并和传统的STRIDE和攻击树建模做比较分析。
    \item 远程实例攻击
    对汽车通过蓝牙劫持信号、数据篡改抓包和车联APP逆向编译等方法进行攻击,为智能网联汽车的安全发展提高新的研究思路。
\end{enumerate}


\section{论文组织结构}
第一章是绪论, 主要介绍的是本文章的研究背景以及意义,对当下智能网联
汽车(ICV)的研究现状进行了介绍和总结,最后说明了本文章节的安排。

第二章是相关理论和基础技术介绍,分别对智能网联汽车的系
统架构、TSP 系统的组成、ICV 通信技术以及智能网联汽车常见的七大安全威
胁做了介绍。对 ICV 的攻击来源做了分析和整理。对目前现有威胁建模常用方法进行了整理和分类、并介绍了模糊层次分析法为后续的威胁模型提出做了前置理论基础准备。

第三章我们首先介绍了对车载IVI系统进行了全面的系统安全分析,并利用传统的STRIDE模型对其进行了威胁建模,最后对几个典型的数据流进行了威胁评级。

第四章我们结合STRIDE和攻击树模型提出一种创新的名为SATT的威胁建模方法,对汽车网络架构进行威胁建模进行了威胁分析、并和传统的STRIDE建模进行了比较分析。

第五章,基于上述提出的威胁建模方法,对汽车通过蓝牙劫持信号、数据篡改抓包和车联APP逆向编译等方法进行实际攻击实验。

第六章,总结与展望。
最后,对全文进行了总结,并对存在的问题和存在的问题进行了分析,并提出了今后的研究方向。
