\chapter{绪\hskip 0.4cm 论}
\label{ch1}
\section{研究背景及意义}
近些年,由于移动网络、物联网以及工业智能化等技术的迅速发展,汽车产
业链也迎来了大变革,汽车由传统的代步工具发展为集出行、娱乐、智能为一
体的行驶在马路上的电脑。汽车的智能和互联性正在推动车载网络系统的复
杂性和多样性的增加。智能网联汽车目前正处于渗透率快速提升的阶段,智能网联汽车能够将手机的百万量级的应用融合到汽车中,
实现汽车应用的量级突破。
智能网联汽车有望继智能手机之后,成为新一代的超级终端。
根据中商情报网数据显示\cite{zhao},2016-2020年我国智能网联汽车产业规模呈现连续上涨趋势,2020年产业规模增长到了2556亿元,同比增长54.3\%。
预计,2022年智能网联汽车产业规模将超3500亿元。
\newline
汽车的智能和互联性正在推动车载网络系统的复
杂性和多样性的增加。信息
通信技术(ICT)给传统车辆带来了巨大的创新,也给现
代车辆带来了重大的安全挑战。原本隔离的车载网络连
接到外部,增加了互联和自动化车辆(CAV)的攻击面,
并带来了新的安全风险。汽车网络安全引起了公众的关
注。
\newline
Checkoway 等人\cite{checkoway2011comprehensive}从网络安全的角度描述了现代车辆
的安全威胁模型和外部攻击媒介,并研究了如何进入车辆
的内部网络。2015 年,查理·米勒和克里斯·瓦拉塞克
(Chris Valasek)演示了如何通过攻击车载信息娱乐(IVI)
系统来入侵一辆吉普车的车载网络,并导致车辆的异常行
为,导致 140 万辆汽车被召回。与此同时,自动网络安全
开始引起公众的广泛关注\cite{miller2015remote}。Koscher 等人 \cite{koscher2010experimental} 通过嗅探
并向控制器局域网(CAN)总线注入恶意代码,通过拒绝服务
(DoS)攻击阻断电子控制单元(ECU)通信,从而控制车辆的
多个功能模块。在\cite{liu2017vehicle}中,针对 CAN 的安全缺陷,提出了广
播、无认证、无加密、仲裁优先、接口易访问等五种对
CAN 的攻击方法。除了通过 CAN 攻击车辆,本地互联网络
(LIN) \cite{deng2017security},FlexRay \cite{takahashi2017automotive} \cite{gu2016security} 总线系统和其他外部接
口,例如通用串行总线(USB),蓝牙,Wi-Fi,蜂窝和车载
诊断(OBD),也可用于非法访问车载网络\cite{mousa2016lightweight}。
\newline
安全威胁远不止前面提到的那些。因此,有必要在
车辆的早期开发阶段进行安全威胁分析,以找出威胁漏
洞,实施安全措施,预防安全问题。在汽车领域有许多
威胁分析和风险评估(TARA)方法。SAE J3061
\cite{sae2016cybersecurity}作为
第一本针对网络物理车辆系统的网络安全指南,提出了
EVITA(电子安全车辆入侵保护应用程序)、HEAVENS (修复漏
洞以增强软件安全性)、TVRA(威胁、漏洞和风险分
析)、OCTAVE(运营关键威胁、资产和漏洞评估)以及其
他安全分析方法。然而,并不是所有提到的方法都适用
于智能网联汽车。TVRA是为数据和电信网络开发的,
不适用于汽车。OCTAVE适用于企业信息安全的风险评
估,但不适用于汽车系统。只有EVITA和HEAVES更适合智能汽
车领域。
\newline
因此,以智能网联信息系统为载体的智能网联汽车,其威胁安全的研究意义将是十分重要和深远的。
\begin{itemize}
  \item 智能网联汽车的信息安全关系到人类的生命和财产的安全,因此,对智能网联车辆的攻击与威胁问题进行深入的研究,可以促进整车厂商对其进行改进,以保障人民的生命和财产的安全。
  \item 通过及时发现智能网联车辆网络系统的缺陷,并对其进行攻击的路径进行描述,可以将智能网联车辆所面临的安全风险进行全面、明确的展现,帮助安全开发者了解到相关知识,并及时定位风险源,从而为进一步研究更安全的防护措施打下基础。
  \item 基于攻击树的威胁建模分析可以建立威胁和安全属性之间的直接映射。它支持更好地理解和列出TOE的威胁。使评价结果更加具有说服力。
\end{itemize}
\section{国内外研究现状}
由于智能网联汽车的高度集成及其智能化,其功能越来越丰富,系统复杂度
也越来越高,对外暴漏的接口也随之增多。从有线网络到无线网络,再到汽车上
的各种插口,汽车与外界联系的手段不断增加,同时网络中将不可避免的存在各
式各样的漏洞,例如软件漏洞、代码漏洞、硬件接口漏洞等,因此,对其攻击的
手段也将变得越来越多样化,对其攻击路径的分析也变得越发艰难,以下列出一
些当前国内外学者研究的比较典型的漏洞利用。
\newline
Checkoway 等人从网络安全的角度描述了现代车辆
的安全威胁模型和外部攻击媒介,并研究了如何进入车辆
的内部网络。他们试图通过系统分析现代汽车的外部攻击面来解
决这个问题。通过广泛的攻击媒介(包括
机械工具、CD 播放器、蓝牙和蜂窝无线电)进行远
程利用是可行的,此外,无线通信信道允许远程车
辆控制、位置跟踪、车内音频泄漏和盗窃。
CAN协议是一种广泛应用于车载网络的总线协议。
CAN 协议有几个内在漏洞,例如广播传输、无认证、无加密、基于 ID 的优先级方案和可用接口。
这些漏洞使车载网络容易受到恶意攻击。Liu等人总结了以下有效攻击方法:帧嗅探,帧伪造,
帧注入,重放攻击,拒绝服务攻击等。此外新型的智能网联汽车具有手机APP等能操控智能网联汽车本身的外部操控智能设备
因此从手机端APP等进行攻击也是目前比较流行的一个新颖攻击方法。
\newline
已有的这些研究表明,智能网联汽车的发展尚未成熟,仍然存在较多的安全隐
患。尤其是针对无线系统的攻击,其隐蔽性强,覆盖范围广,往往会给驾驶者带来
严重的安全隐患和灾难性后果。当前对于智能网联汽车的安全意识正在逐步增强,
但是并没有形成统一的理论体系和安全标准,智能网联汽车的安全问题不仅仅牵涉
到经济问题,还关系驾驶者和其他人的生命安全。因此,研究整个智能网联汽车的
信息安全至关重要.
\section{本文工作和主要贡献}
\paragraph{本文贡献点主要有如下三个:}

\begin{enumerate}
    \item 智能网联汽车架构及威胁分析 
    分析智能网联汽车的系统架构及其通信机
    制,并从云平台、APP、T-BOX、IVI、CAN 总线、ECU、车间通信七个方
    面分析智能网联汽车所面临的安全威胁。
    \item 分析了TARA中主流的威胁建模方法如HEAVENS, EVITAS, STRIDE, 攻击树等。
    研究智能网联汽车的威胁模型,结合STRIDE和攻击树模型提出一种创新的名为SATT的威胁建
模方法,对实际商用的汽车外部复杂网络中的通信和TSP部分进行威胁建模,并对
两种建模方法进行比较和分析。
    \item 风险评估与威胁框架
    研究复杂网络环境下智能网联汽车的远程攻击方法,对某知名品牌汽车进
行漏洞挖掘和渗透测试,利用硬件拆检、协议分析、算法破解和软件逆向等多种方
式分析漏洞,并编写软件,成功获取目标车辆敏感信息,并实现了远程入侵和完全
控制,所阐述的一系列攻击方法在智能网联车领域又是一种全新
的攻击手段,为智能网联汽车安全研究提供更多的思考空间和方向,为智能网联汽
车的安全发展奠定基石。

\end{enumerate}


\section{论文组织结构}
本文主要研究针对智能网联汽车领域的威胁分析与风险评估。是在车联网概念阶段应用的一种分析技术,可帮助识别特征的潜在威胁 并评估与已识别威胁相关的风险。
具体章节结构如下:

第\ref{ch1}章,绪论。
主要介绍的是本文章的研究背景以及意义,对当下智能网联
汽车(ICV)的信息安全及风险评估领域的研究进行了介绍和总结,并对文章的
主要工作和文章的章节进行了介绍。

第\ref{ch2}章,相关概念及研究。
对本文中的相关理论与技术基础进行了详细介绍,分别对 ICV 的系
统架构、TSP 系统的组成、ICV 通信技术以及智能网联汽车常见的七大安全威
胁做了介绍。
对 ICV 的攻击来源做了分
析和整理。对目前现有TARA常用方法进行了整理和分类。

第\ref{ch3}章
研究智能网联汽车的威胁模型,介绍了传统的 STRIDE 威胁建模方法尝建模分析和攻击树模型。

第\ref{ch4}章,
研究智能网联汽车的威胁模型,结合STRIDE和攻击树模型提出一种创新的名为SATT的威胁建
模方法,对实际商用的汽车外部复杂网络中的通信和TSP部分进行威胁建模。

第\ref{ch5}章,基于上述提出的威胁建模方法
研究复杂网络环境下智能网联汽车的远程攻击方法,对某知名品牌汽车进
行漏洞挖掘和渗透测试,利用硬件拆检、协议分析、算法破解和软件逆向等多种方
式分析漏洞,并编写程序,成功获取目标车辆敏感信息,并实现了远程入侵和完全
控制。

第\ref{ch6}章,总结与展望。
对本文工作进行总结,分析本文的不足以及尚未解决的问题,同时对下一步工作进行展望。
