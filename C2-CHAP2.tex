\chapter{相关概念及研究}
\label{ch2}

本章首先介绍了包括TSP云服务、V2X车载通信技术等在内的智能网联汽车的网络系统架构,对其系统架构的构成及原理分别进行了分析,其次对当前智能网联汽车面临的主要攻击手段及主要威胁进行了介绍。

\section{智能网联汽车联网系统架构}

\subsection{车载网络架构划分}
车载网络架构如图2.1所示。这是一个
以域为中心的体系结构,具有以太网作为主干,分为五个领域:动力系统、底盘、车
身、信息娱乐和高级驾驶辅助系统(ADAS)。每个域控制
器通过一个中央网关与以太网主干连接。CAN/CAN FD 和
LIN 用作每个域中的通信协议。此外,车载网络可以通
过远程信息处理单元和接口(如 OBD、USB 和 Wi-Fi)连接
到外部网络。这种集中式架构通过域控制器和以太网提
供智能网联汽车所需的计算和通信能力。然而,车辆的外部接
口也增加了。这些开放的接口给智能网联汽车带来了新的攻击
面和安全隐患。
\begin{figure}
  \centering
  \includegraphics[scale=0.4]{resources/img/i1.jpg}
  \caption{车联网网络架构图}
\end{figure}
现在的汽车结构中,主要由CAN总线和ECU构成的车辆内部网络。ECU是嵌入式设备,包含了紧急制动辅助(EBA,ElectronicBrakeAssist)、防抱死制动系统(ABS,ANTILOCKBRAKESYSTEM)等多种智能系统。CAN总线是 ECU间的接口
能把汽车里的 ECU 连接在一起的通讯桥,
使它们能够进行高效的信息通信。

在使用 CAN总线时,由于其与大量的内嵌式设备之间的广泛联系,给使用者带来了极大的便利,但是也带来了一定的安全威胁。
黑客能够利用车载信息娱乐(IVI,in-vehicle infotainment)系统暴露出来的USB等接口恶意进入系统内部植入恶意软件获取隐私信息。

车载诊断界面主要用于车辆故障的检测和废气监测,而OBD-II接口则一般是非法侵入车辆 CAN总线数据的通道。另外,T-BOX,胎压。
许多嵌入式设备如 TPMS、 RKE、 ABS等都有大量可利用的攻击面。

这里简单的介绍上述英文名词的定义。
\begin{itemize}
    \item CAN 控制器局域网(Controller Area Network,简称CAN或者CAN bus) CAN 总线是一种基于消息的协议,旨在允许当今汽车以及其他设备中的电子控制单元 (ECU) 以可靠的、优先级驱动的方式相互通信。网络中的所有设备都可以接收消息或“帧”,这不需要主机。CAN 受到ISO-11898下的一组丰富的国际标准的支持。
    \item ADAS 可帮助驾驶员进行驾驶和停车功能。通过安全的人机界面,ADAS 提高了汽车和道路的安全性。ADAS 使用传感器和摄像头等自动化技术来检测附近的障碍物或驾驶员错误,并做出相应的反应。ADAS 可以实现不同级别的自动驾驶,具体取决于车内安装的功能。
    \item OBD是英文On-Board Diagnostics的缩写,是智能网联汽车内部的计算机系统,用于跟踪和调节汽车的性能。该车载计算机系统从车辆内部的传感器网络收集信息,然后系统可以使用这些信息来调节汽车系统或提醒用户注意问题。技术人员可以简单地插入 OBD 系统以收集车辆数据并诊断问题。OBD 系统在帮助用户更好地了解车辆诊断方面提供了很大帮助。
    \item ECU(Electronic Control Unit)电子控制器单元,又称为汽车的“行车电脑”,是车辆内部的一种小型设备,用于控制该车辆中的一个或多个电气系统。它告诉电气系统该做什么以及如何操作。ECU的核心是一个微控制器,由嵌入式软件控制。它们用于控制汽车的汽车的行驶状态。它主要是利用各种传感器和总线进行数据的收集和交换,从而判断出驾驶员的驾驶意图和驾驶意图,并由执行器控制汽车。
    \item T-BOX作为无线网关,是远程信息处理控制单元(TCU),由GPS单元、通信外部接口、电子处理单元、微控制器、移动通信单元和内存组成,实现车内终端信息的交互、云和路边单元(RSU)。一般具有故障诊断、低功耗设计、休眠唤醒、SD卡扩展、输出电源、RS232/485、USB和IO接口等新功能。
\end{itemize}
\subsection{TSP 云端通信技术}

(Telematics Service Provider)汽车远程服务提供商,其主要提供基于互联网和移动通信技术的远程车辆监控、远程控制、车辆数据分析和诊断等服务。这些服务可以使车主通过智能手机或其他互联网设备,实现远程控制车辆的功能,例如开启或关闭车门、启动或停止发动机、调整车内温度、锁定或解锁车辆等,同时还能够通过实时监测车辆的状态、行驶路线、油耗、故障码等数据,提供更为精准的车辆管理和维护服务,从而提高车主的使用体验和安全性能。汽车远程服务提供商的出现,为汽车行业注入了新的活力和创新动力,推动了汽车行业的数字化转型和服务升级。
 \begin{figure}
    \centering
    \includegraphics[scale=0.6]{resources/img/i2.png}
    \caption{TSP 系统组成}
  \end{figure}

  远程信息处理系统通常具有以下组件:
  \begin{itemize}
    \item 车队通信软件系统
    \item GPS跟踪设备
    \item 引擎接口
    \item 输入/输出接口
    \item SIM卡
    \item 加速度计
    \item 蜂鸣器
  \end{itemize}

  根据其业务模式和服务内容的不同,可以将汽车远程服务提供商分为以下几种类型:

  1.汽车制造商自营型远程服务提供商:由汽车制造商自己开发和提供远程服务平台,向车主提供车辆远程监控、控制、故障诊断和维护等服务。例如,福特的“福特Pass”和特斯拉的“特斯拉网络”等。
  
  2.第三方平台型远程服务提供商:由独立的第三方公司开发和提供远程服务平台,向多个汽车制造商和车主提供服务。例如,OnStar和UVO等。
  
  3.车载通信设备厂商型远程服务提供商:提供车载通信设备和相关服务,以支持车辆远程监控、控制和数据传输等功能。例如,华为、高通等。
  
  4.车联网解决方案提供商型远程服务提供商:提供车联网解决方案和服务,以支持车辆远程监控、控制、故障诊断和维护等功能。例如,百度车联网、车易通等。
  
  以上是一些常见的汽车远程服务提供商类型,随着技术和市场的不断发展,还可能出现新的类型。

TSP一词狭义的在互联汽车行业中被用作
对服务提供者进行分类的广义术语
以安全车到云为核心的汽车价值链
数据管理。然而,目前TSP扮演的传统角色
在价值链中不断进化。TSP一词已被
IT公司,系统集成商,甚至一级企业等采用。网络
运营商正在将其M2M/IOT服务扩展到
汽车行业,意图将数据连接“去商品化”。
越来越多的汽车制造商通过TSP创造和集成更多的车载部件。

\subsection{V2X 车载通信技术}

Vehicle to Everything (V2X) 是一种车载通信系统,支持将信息从车辆传输到可能影响车辆的交通系统的移动部件。V2X 技术的主要目的是提高道路安全、节能和道路交通效率。

车联网的工作原理: 在 V2X 通信系统中,信息通过高带宽、高可靠性的链路从车辆传感器和其他来源传播,使其能够与其他汽车、停车位和交通信号灯等基础设施以及使用智能手机的行人进行通信。
通过与车辆周围的其他实体共享速度等信息,该技术提高了驾驶员对潜在危险的认识,并有助于降低伤害、道路事故死亡和与其他车辆碰撞的严重程度。
该技术还通过警告驾驶员即将到来的交通、建议替代路线以避免交通和识别可用停车位来提高交通效率。

V2X架构主要为V2V(车对车)和V2I(车对基础设施)。V2V 允许车辆与道路上的其他车辆进行通信,而 V2I 允许车辆与外部实体进行通信,例如交通信号灯、停车位、骑自行车的人和行人。这些技术有助于改善道路安全、减少燃料消耗并增强驾驶员与其他道路使用者(例如骑自行车者和行人)之间的体验。

当 V2X 系统集成到传统车辆中时,驾驶员可以接收有关天气模式、附近事故、道路状况、道路工程警告、紧急车辆接近以及使用同一条道路的其他驾驶员活动的重要信息。

配备 V2X 系统的自动驾驶汽车可以为车辆现有的导航系统提供更多信息。该系统还使自动驾驶汽车能够扫描周围环境并根据收到的信息立即做出决定。
\begin{figure}
    \centering
    \includegraphics[scale=0.6]{resources/img/i3.png}
    \caption{v2x 应用场景}
  \end{figure}
如图2.3所示,智能网联汽车通过车辆间通信(V2X)实现了车辆之间、车辆与基础设施之间的信息交换和协同,包括短程通信(DSRC)和移动蜂窝网络(C-V2X)等技术。这些技术允许车辆获取周围环境信息和其他车辆的动态行驶信息,利用高精度地图、车辆传感器和自主控制系统等技术实现自动驾驶、避免碰撞和协同行驶等功能,从而提高交通安全和效率。同时,智能网联汽车还可以与云端通信,获取更丰富的数据和服务支持,实现更加精准的路线规划和出行服务,从而提高用户体验和生活品质。
\section{智能网联汽车主要攻击手段}
发明车载网络协议时,安全问题并不是主要问题。因此,许多安全功能天生就缺失了。例如,CAN 缺乏必要的保护来确保信号的可用性、机密性和真实性\cite{woo2014practical}。
FlexRay 虽然能够在出现错误的情况下保持正确操作,但无法抵御格式良好的恶意错误消息\cite{kleberger2011security}。尽管如此,这些缺点在过去并未构成迫在眉睫的安全威胁,因为车辆很少与外界连接,
而老式的安全攻击通常需要对车载网络进行物理访问。/

然而,现代车辆正在通过各种方式迅速变得更加互联,用于许多高级应用。例如,车辆可以通过 DSRC(专用短程通信)连接以实现 VANET(车载自组织网络)功能,
通过 Wi-Fi/蓝牙实现车载娱乐,并通过蜂窝网络实现远程信息处理服务。尽管这些连接使车辆更加智能和舒适,但它们也将车载网络大量暴露给外部对手。
例如,CAN 通信可能会被智能手机恶意软件通过蜂窝网络远程篡改\cite{woo2014practical}。软件病毒可能通过受感染的娱乐媒体(如 CD(光盘)或蓝牙播放器)传播到车载组件。
此外,还可以通过攻击OEM存储中心的ECU密钥管理不善来侵入车载网络。

此外,直接访问的威胁仍然存在,因为攻击者也可能物理侵入通信线路,直接针对网络组件的弱点发起攻击。这种典型的攻击可能包括反汇编可执行代码和将恶意代码注入运行时环境。

车载网络的安全漏洞不仅可能对车辆用户造成严重后果,还会对其他道路交通参与者造成严重后果。例如,安全漏洞可能导致车辆用户的隐私泄露。目标私人数据可能包括车辆诊断流
、机柜对话、摄像记录、驾驶模式和车辆位置\cite{amoozadeh2015security}. 
这种典型的攻击是通过未经授权的窃听进行的。其次,安全漏洞可能导致车主或原始设备制造商的直接金钱损失。在此类攻击中,
攻击者经常故意修改或重放所需的车载数据以实现非法收益,例如车辆盗窃或里程表欺诈。第三级安全漏洞可能会对车辆使用者造成安全威胁。
这种攻击通常涉及对安全关键车载数据的恶意修改或伪造,例如轮胎压力、车速、发动机扭矩请求和制动命令. 
这可能导致非自愿驾驶机动甚至交通事故。考虑到自动驾驶的出现,这种危险至关重要,值得研究界更多关注。
第四,车载网络安全漏洞会对其他道路参与者造成安全威胁,甚至瘫痪整个交通系统。由于车辆将在大型网络中互连,
例如 VANET,因此信号可信度对于协调交通系统中的所有车辆都极为重要\cite{harding2014vehicle}.
 但是,如果车载网络安全受到损害,这种可信度可能会被破坏。
 例如,被篡改的车载网络可能会产生虚假数据,如果虚假数据已经传播到车辆外部并被其他人认为是“值得信赖的”,则可能对其他车辆造成极大的危险。
综合上述研究现状,将攻击手段分为以下类型:
\begin{itemize}
    \item 远距离通信攻击: 如利用蜂窝网络、Wi-Fi等进行伪装拦截通信信号等从而达到攻击的目的。
    \item 近距离车外通信: 利用蓝牙攻击和高频无线电攻击。如通过蓝牙连接车载娱乐系统,伪装发送信号给车载娱乐系统从而达到攻击的目的。
    \item 车辆内部网络: 如通过车辆内部USB攻击IVI系统等。
\end{itemize}

\section{智能网联汽车面临的安全威胁}
安全性是智能网联汽车面临的迅速出现的重大挑战。在车载网络的背景下,安全问题通常是指通信数据可能被恶意攻击者窃听、欺骗、丢弃、修改、泛滥、窃取等危险情况。
在车辆向自动驾驶和协作驾驶发展的时代,安全性在车载网络的设计中变得越来越重要。

\subsection{ICV 中的潜在威胁}
在远距离通信中,恶意攻击者入侵汽车的方式一般可分为:蜂窝网络、Wi-Fi、车载单元
(OBU,on board unit)/路侧单元(RSU,road side
unit)和全球定位系统(GPS,global positioning
system)等四大类

(1) 蜂窝网络
蜂窝网解决了 ICV的远程通讯问题,但也存在着一些新的安全问题。比如,
利用无线通讯通道进行汽车定位追踪及通讯监视。
破解汽车固件,可实现了车辆的遥控(如方向盘等)。


(2) Wi-Fi
黑客通过 Wi-Fi 连接可以进行很多恶意攻击。如利用 Wi-Fi 远程访问车内网络;入侵IVI系统安装木马;监控汽车通信管道和访问流量。在文献\cite{keen}中,腾讯科恩安全实
验室研究员远程入侵了特斯拉汽车的网关、BCM 和
自动驾驶系统。它还能实现特斯拉电动汽车的天窗和车门的遥控启动,以及在驾驶过程中制动
利用安全漏洞,远程入侵了特斯拉车的网络。
并且可以对特斯拉进行任意的车身和驾驶控制。

(3) GPS
GPS(全球卫星定位系统)是由美国政府开发和维护的卫星导航系统,主要由一组卫星和地面控制设施组成。GPS的功能是通过卫星信号为用户提供准确的位置、速度和时间信息能够为汽
车提供最优的行车路径,从而确保 GPS系统的安全性。
文献\cite{cuigai}
显示了使用便携 GPS诱骗装置对汽车GPS信号进行了篡改,从而达到远程控制汽车的目的。

\subsection{近距离车外通信的潜在威胁}
在近距离的车外通讯中,主要有以下几种方式可以让恶意攻击者入侵汽车:RFID(无线射频识别)攻击,NFC(近场通讯)攻击,蓝牙攻击,远程钥匙攻击。

(1) 身份识别攻击:利用近距离通信技术(如NFC、蓝牙、Wi-Fi Direct等)在车辆周围进行身份识别攻击,从而欺骗车辆或者远程服务器,获取敏感信息或者非法控制车辆。

(2) 入侵车载系统攻击:在车辆附近或车内,通过近距离通信技术进行入侵车载系统,从而获取车辆信息或非法控制车辆。例如,通过蓝牙连接入侵车载系统,获取车辆的行驶记录、地理位置等敏感信息。

(3) 物理攻击:通过直接接触车载设备进行攻击,例如通过物理接触、拆卸等方式攻击车辆ECU(发动机控制单元)等。

(4) 数据篡改:对车辆的数据进行篡改,例如篡改车辆传感器的数据、车辆位置信息等,以影响车辆的正常运行。

以上是近距离通信的潜在威胁,网络安全公司和汽车制造商需要采取相应的安全措施,以保护智能网联汽车的安全性能。例如,通过加密和认证技术,确保车辆和服务器之间的通信安全;通过网络安全防护技术,确保车辆内部系统的安全性能等。

\subsection{车辆内部网络的潜在威胁}
(1) 内部攻击:利用车载系统内部的漏洞或者恶意软件,攻击车辆内部网络或者其他车载系统,获取敏感信息或者非法控制车辆。例如,通过攻击车辆的CAN总线,控制车辆的刹车或者加速器等关键系统。文献\cite{koscher2010experimental}利用侧信
道攻击,通过CAN总线采集数据通信来窃听驾驶
人员的个人隐私,证实ICV中存在用户的个人隐私信息
有被泄漏的危险。

(2) 外部攻击:利用外部设备或者无线信号,攻击车载系统或者车辆内部网络。例如,通过无线电波攻击车载系统,篡改车辆传感器的数据或者获取车辆敏感信息。

(3) 物理攻击:通过直接接触车载设备进行攻击,例如通过物理接触、拆卸等方式攻击车辆ECU(发动机控制单元)等。

(4) 软件漏洞:车辆内部系统中可能存在软件漏洞,黑客可以利用这些漏洞进行攻击,从而获取车辆敏感信息或者非法控制车辆。

为了保护车辆内部网络的安全性能,汽车制造商和网络安全公司需要采取相应的安全措施。例如,通过加密和认证技术,确保车辆内部系统之间的通信安全;通过网络安全防护技术,防止恶意软件入侵车辆内部系统等。此外,汽车制造商需要在车辆生产过程中,注重软件安全性能的检测和修复,以降低车辆内部系统被攻击的风险。


\section{STRIDE威胁建模分析}
STRIDE是一个信息安全威胁建模框架,它被用于识别和评估软件系统中的安全威胁。  
\subsection{四类元素和六大威胁}

STRIDE安全威胁建模方法是将威胁从攻击者的角度分成6类,分别是Spooling(仿冒)、Tampering(篡改)、Repudiation(抵赖)、Information Disclosure(信息泄露)、Denial of service(拒绝服务)和Elevation of privilege (权限提升)。
每种威胁都代表着一种攻击者可以利用的不同技术或策略,攻击者可以利用这些技术或策略来利用系统的漏洞和缺陷,从而获得不当的访问权限或控制权。STRIDE框架的目的是帮助软件开发人员、安全工程师和系统管理员在设计、实施和维护软件系统时识别和处理安全威胁。通过识别系统中存在的安全威胁,可以更好地保护软件系统免受未经授权的访问、数据泄露、服务中断等安全威胁的影响。

建立 STRIDE的威胁模型,首先要画出数据流,数据流是由四大要素构成,分别是:【外部实体】、【处理过程】、【数据存储】和【数据流】。
STRIDE威胁模型的关键在于利用四种不同类型的要素来绘制数据流图,并根据不同的威胁类型,对不同类型的威胁进行相应的消除。


四类元素的介绍如下:

(1) 外部实体:
用户、软件系统或系统不受系统控制的装置。一种系统或一种产物的输入或输出。将外部实体以一个长方形的形式表示在数据流图中。

(2) 处理过程:
表明一项任务,一项实施过程,其中必然有流入和流出的数据。图中用圆形表示的数据流图

(3) 数据存储:
数据库,消息队列,文件等存储数据的内部实体,用中间带标签的两条平行线表示。

(4) 数据流:
表示数据流转的是外部实体和过程、过程和过程或过程和数据存储之间的交互。用箭头表示在数据流图表中。

使用以上四个元素绘制完数据流图后,还需要引入信任边界,信任边界(在威胁建模的上下文中)是数据流图中数据更改其信任级别的位置。数据在两个进程之间传递的任何地方通常都是信任边界。信任边界一般是攻击者喜欢发起的地方。在数据流图中可以用红色的虚线隔离出信任边界。

% 图2.4是一个比较简单的数据流图演示:
% \begin{figure}
%     \centering
%     \includegraphics[scale=0.6]{resources/img/i5.png}
%     \caption{简单数据流图实例}
%   \end{figure}


\subsection{实体与威胁的对应关系}

具体的对应关系如图2.4,举例来说,如数据流只有篡改、信息泄漏、拒绝服务三类威胁,其他剩下的三类威胁永远不会发生在数据流中。
又如进程会面临全部的6个威胁,数据存储中Repudiation(抵赖)是红色,表示只有存储的数据是审计类日志才会有抵赖的风险。
\begin{figure}
    \centering
    \includegraphics[scale=0.6]{resources/img/i77.png}
    \caption{四类元素和六类威胁对应关系}
  \end{figure}

\subsection{STRIDE 威胁建模流程}
(1) 分解业务场景,绘制数据流图表(DFD),以特定的场景为目标,
因此,应当按照实际的应用情况,如电商业务中采购、登记等进行分类。如果有
有多少种情况,或者如何划分,取决于用户所用的系统和企业。淘宝,京
与电商平台如拼多多、 AmazonEC2、腾讯等国产平台。不同的系统会产生不同的商业场景。每个服务与每个场景
STRIDE威胁建模是独立的,不会互相干扰,所以每个已分解的业务场景都要进行深入的研究,
对不同的 STRIDE威胁进行建模。接下来,就是要画出一个数据流图表。数据流图
四个单元:数据流(箭头)、数据存储单元(双横线)、进程(圆形)以及外部实体/交互方(正方
素数的标准符号。在图2.5中可以看到。
除以上四大核心部件之外,在实践中,还可以添加一个叫做“信任边界”的元素,我们用虚线标出的该元素。当数据流通过不同的信任界限时,就必须使用
用线条来区别信任边界。因为在接下来的威胁模型中,信任边界是不会被划分
所以,大部分的核心要素都包含:四大元素:进程,数据存储,数据流,以及交互方/外部实体
这里没有所谓的“信任边界”。在增加了信任边界之后,每个节点和每个数据流都必须执行分析
对其在6个维度的安全威胁进行分析和判定。

\begin{figure}
    \centering
    \includegraphics[scale=0.4]{resources/img/i7.jpg}
    \caption{常用web场景数据流图}
  \end{figure}
  (2)为了确保系统安全性,需要在绘制完数据流图后,对数据流中的每个节点元素进行安全威胁分析。同时列出每个节点元素可能面临的安全威胁,并逐一作详细全面的分析。如表 2.1 所示。

  表中列出了每种元素可能面临的安全威胁,下面对表进行简单分析数据流需要分析“篡改(替换)、外部实体可能面临“伪装”和
  “抵赖”两种威胁、“信息丢失或泄露”和“拒绝服务”三种威胁。

  \begin{table}
    \caption{元素STRIDE威胁分析}
  \begin{center}
    \begin{tabular}{|l|l|l|l|l|l|l}
      \hline 元素类型 & 伪装 & 篡改 & 抵赖 & 信息泄露 & 拒绝服务 & 权限提升\\
      \hline 外部实体 & X & X & X & X & X & X \\
      \hline 进程 & X & X & X & X & X & X \\
      \hline 数据存储 & X & X & X & X & X & X \\
      \hline 数据流 & X & X & X & X & X & X \\
      \hline
      \end{tabular}
  \end{center}
\end{table}
  (3)输出威胁列表:
  对数据流图中的全部元素进行了安全威胁性分析后,输出了威胁列表如表2.2所示:包括消减方案和威胁评级。威胁列表可以帮助生成我们对威胁进行深入分析和改进的模型。
\begin{table}
  \caption{威胁列表}
\begin{center}
  \begin{tabular}{|l|l|}
    \hline 组件(威胁的目标) & GPS自动导航进程 \\
    \hline 威胁描述 & 黑客通过劫持GPS信号误导用户自动导航 \\
    \hline 威胁类别 & I \\
    \hline 攻击方法 & 利用网络监控劫持信号通道 \\
    \hline 消减方案(对策) & 提供加密和验证通道 \\
    \hline 危险评级 & 待定 \\
    \hline
    \end{tabular}
\end{center}
\end{table}
威胁等级,基于所产生的后果来评估和评分。这样就能对威胁进行归类,
首先要处理的是最大的威胁,其次才是较小的。学术上目前有很多种不同的威胁等级。
以安全缺陷等级为例,它包括 DREAD和 CVSS (Common Vulnerability Scoring System)
两种常见评级方法。不同的评估方式在维度和计算上存在细微差
不同,但从本质上讲,危险等级=发生的几率×所造成的损失。具体要根据实际用途使用。

DREAD是一个常用的安全风险评估框架,它由五个因素组成,分别是:Damage(损害)、Reproducibility(再现性)、Exploitability(利用难度)、Affected Users(受影响的用户)和Discoverability(可发现性)。

每个因素都有一个分值(通常是0到10),分值越高表示该因素对系统安全的威胁越大。安全评估人员可以通过对每个因素进行分析和评估,计算出系统的总体安全风险分值。DREAD框架的目的是帮助安全评估人员、开发人员和系统管理员更好地了解系统中存在的安全威胁,并采取相应的措施来降低风险。

最后,我们给出的威胁等级评级公式是通过以下6个指数进行加权平均值得到的,见公式2.1:
\begin{equation}
\operatorname{Rank}[0: 10]=\frac{(D a m+\operatorname{Rep}+\operatorname{Exp}+A f f+\text { Dis })}{2}
\end{equation}

Rank代表着威胁等级,从0到10,代表着从零到高的危险。这一公式并非统一不变的标准,只是为了让该评级方式的威胁值与 CVSS范围保持相同。
每个指标的得分为0至4,表明了威胁的严重性。

\begin{table}
  \caption{威胁评级}
\begin{center}
  \begin{tabular}{|l|l|l|l|l|}
    \hline 等级 & 极高(4) & 高(3) & 中(2) & 低(1) \\
    \hline 潜在的损失D & 获取最高权限 & 泄露关键信息 & 泄露敏感信息 & 泄露其他信息 \\
    \hline 重现性R & 可随时攻击 & 易重复攻击 & 可重复攻击 & 难重复攻击 \\
    \hline 可利用性E & 非常容易利用 & 较为容易攻击 & 高级黑客可利用 & 攻击条件苛刻 \\
    \hline 受影响用户A & 所有用户 & 管理员 & 一般用户 & 其他用户 \\
    \hline 可发现性D & 漏洞过于明显 & 需要漏洞挖掘 & 限定范围可发现 & 隐藏性极高 \\
    \hline
    \end{tabular}
\end{center}
\end{table}




    

\section{本章小结}

本章从智能网联汽车的组成架构开始分析,其次对智能网联汽车的 TSP 云服务、V2X,车载网络通信进行了详细分析。
接着对三种不同的威胁进行了描述。本文重点对 STRIDE系统进行了威胁模型的分析。为后续的智能互联汽车威胁模型的提出提供了前置理论来源。