\newpage
\vspace{-1cm}
\chapter*{\xiaosan \heiti{ABSTRACT}}
\addcontentsline{toc}{chapter}{Abstract}
\vspace{-0.5cm}
\hspace{-0.5cm}
With the rapid development of industrial intelligence, big data, mobile Internet, Internet of Things, cloud computing and other technologies,
In a brand-new way, the automobile is changing from a traditional closed system to an open system which integrates new concepts such as smart city, intelligent transportation, intelligent communication and intelligent driving.
Modern intelligent networked cars have many sensors, and various internal circuits connect and fuse electronic control units and sensors, thus forming an automobile internet network containing millions of lines of codes.
Therefore, the complex network wiring system, various electronic control units, sensor communication, etc. will all face unprecedented security threats and attacks,
The problem of automobile network security is becoming more prominent than ever before. SAE J3061 \cite{schmittner2016using} and ISO/SAE 21434 \cite{schneider2022iso}, which are being drafted, also show that automobile network security has been raised.
Rise to the same or more important position as functional safety.
This paper has the following three main contributions:
\begin{itemize}
    \item Firstly, the architecture of intelligent networked vehicles is studied, and the security threats it faces are analyzed, and the attack sources and attack paths are analyzed.
    \item Not only the traditional attack tree is used to model the threat of intelligent networked automobile system, but also an improved attack tree threat modeling scheme based on Microsoft STRIDE (called SATT) is proposed.;
    \item SATT is applied to the complex network attack path of intelligent networked vehicles, and the model is built. The threat of the attack tree is modeled, and the attack probability is calculated based on FAHP (Fuzzy Analytic Hierarchy Process), and the usability of the model is verified.。
  \end{itemize}
  This paper aims to make more scholars pay attention to this field through the research of intelligent networked vehicle threat modeling and safety risk assessment, so as to promote the development of intelligent networked vehicle threat safety modeling and safety risk assessment.
  Traditional threat analysis based on attack tree has many subjective factors and low accuracy. This paper presents a new threat modeling scheme of SATT.
  With the help of Microsoft's STRIDE threat model, we identify threats corresponding to assets and build a more comprehensive attack tree. most
  Then, for each threat attack sequence in the attack tree, the attack probability is calculated based on FAHP. In addition, FAHP is combined with analytic hierarchy process and
  The traditional methods are compared. Finally, a self-driving experimental vehicle was attacked by network.
  Path analysis and modeling experiments are carried out, and the security risk assessment of the model based on the threat model is carried out. Finally, the experiment
  The evaluation data are analyzed, the usability of the model is verified, and good results are achieved.
  The results show that this method plays a certain role in security threat analysis.
  \newline
{\sihao{\textbf{Keywords:}}}\textit{ICV; TARA; STRIDE; ATTACK TREE; SATT}