\newpage
\vspace{-1cm}
\chapter*{\xiaosan \heiti{ABSTRACT}}
\addcontentsline{toc}{chapter}{Abstract}
\vspace{-0.5cm}
\hspace{-0.5cm}
With the rapid progress of industrial intelligence, big data, the Internet of Things and cloud computing,
In a new way, the car is shifting from traditional closed systems to smart cities, intelligent transportation, intelligent communication, and intelligent driving. Therefore, modern intelligent connected cars will face unprecedented security threats and attacks. Hackers or attackers can access and control vehicles illegally through external interfaces such as Bluetooth, Wi-Fi and honeycomb. Automotive network security issues are becoming more prominent than ever. More and more international security standards have also shown that automobile network security has been improved to the same or more important position as functional security. This article mainly has the following three contributions:
\begin{itemize}
    \item Firstly, the architecture of intelligent networked vehicles is studied, and the security threats it faces are analyzed, and the attack sources and attack paths are analyzed.
    \item Not only the traditional attack tree is used to model the threat of intelligent networked automobile system, but also an improved attack tree threat modeling scheme based on Microsoft STRIDE (called SATT) is proposed.;
    \item The simulation and modeling experiments were performed through the well -known security software ISOGRAPH Attacktree +. Study the remote attack method of intelligent connected cars, conduct vulnerability mining and penetration tests for cars, attack the vehicle system through various methods such as Bluetooth hijacking signals, data tampering bags and car union APP cracking, and successfully obtain target vehicle sensitive information and real -time location.  And achieve remote invasion and fully control
  \end{itemize}
  It aims to provide more risk assessment and mitigation personnel with the research on intelligent network vehicle threat modeling.
  For a more comprehensive security risk assessment. This paper proposes a novel threat modeling scheme for SATT. Experimental results show that,
  This method has played a certain role in security threat analysis. For cars, we conducted a real car attack experiment
  test. Using Bluetooth to hijack signals, data tampering and packet capture, and reverse compilation to crack the Autolink APP. These attacks
  An attack can cause the car to be controlled remotely, posing a threat to the life safety of the intelligent networked car and the driver.
  \newline
{\sihao{\textbf{Keywords:}}}\textit{ICV; TARA; STRIDE; ATTACK TREE; SATT}