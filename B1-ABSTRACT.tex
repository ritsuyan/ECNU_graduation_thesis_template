\vspace{-2.5cm}
\chapter*{\xiaosan \heiti{摘~~~~要}}
\addcontentsline{toc}{chapter}{摘要}
\hspace{-0.5cm}

随着工业智能、大数据、物联网和云计算等技术的快速进步,
汽车正以一种全新的方式,从传统的封闭系统转向智能城市、智能交通、智能通信、智能驾驶等新的概念融合的开放系统。因此,现代智能网联汽车将面临前所未有的安全威胁和攻击。黑客或攻击者可以通过蓝牙、Wi-Fi和蜂窝等外部接口非法访问和控制车辆。汽车网络安全问题正变得比以往任何时候都更加突出。越来越多的国际安全标准也表明,汽车网络安全已经被提升到与功能安全同等或更重要的地位。本文主要有以下三个贡献点:
\begin{itemize}
    \item \textbf{对车载IVI系统进行了全面的系统安全分析,并利用传统的STRIDE模型对其进行了威胁建模,最后对几个典型的数据流进行了威胁评级。}
    \item \textbf{提出了一种基于STRIDE模型的改进后的攻击树威胁建模方案(称为SATT),将SATT应用在实际智能网联汽车网络,建立了威胁分析模型,并利用FAHP(模糊层次分析法)对攻击概率进行了计算,进行了风险评估,最后验证了模型的可用性。}
    \item \textbf{通过知名安全软件Isograph AttackTree + 进行了仿真建模实验。研究智能网联汽车的远程攻击方法,对汽车进行漏洞挖掘和渗透测试,通过蓝牙劫持信号、数据篡改抓包和车联APP破解等多种方式攻击车载系统,成功获取目标车辆敏感信息和实时位置,并实现了远程入侵和完全控制。}    
\end{itemize}

旨在通过对智能网联汽车威胁建模的研究,为更多的风险评估和缓解人员提供更全面的安全风险评估。
基于传统攻击树的威胁分析存在主观因素多、准确率低的缺点。本文提出了SATT新型威胁建模方案。借助微软的STRIDE威胁模型,我们识别与资产相对应的威胁,并构建一个更全面的攻击树。对于攻击树的每个威胁攻击序列,基于FAHP计算攻击概率。此外,将其与传统方法进行了比较。结果表明,该方法在安全威胁分析中起到了一定的作用。最后,针对汽车,黑客可以使用多种方法进行攻击,比如利用蓝牙劫持信号、数据篡改抓包以及反向编译破解车联APP等方式。这些攻击可以导致汽车被远程控制,从而对智能网联汽车和驾驶者的生命安全造成威胁。
\newline
\sihao{\heiti{关键词:}} \xiaosi{智能网联汽车,风险评估,威胁建模,STRIDE, 攻击树, SATT}