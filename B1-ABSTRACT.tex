\vspace{-2.5cm}
\chapter*{\xiaosan \heiti{摘~~~~要}}
\addcontentsline{toc}{chapter}{摘要}
\hspace{-0.5cm}
随着工业智能、大数据、移动互联网、物联网、云计算等技术的飞速发展,
汽车正以一种全新的方式,从传统的封闭系统,转向智能城市,智能交通,智能通信,智能驾驶等新的概念融合的开放系统。
现代智能网联汽车拥有许多感应器,其内部的各种线路将电子控制单元和传感器彼此相连、融合,构成了一个包含几百万行代码的汽车互联网络。
因而,复杂的网络配线系统、各类电子控制单元、传感器通信等都将面临前所未有的安全威胁和攻击,
汽车网络安全问题正变得比以往任何时候都更加突出。正在起草的SAE J3061\cite{schmittner2016using} 和 ISO/SAE 21434 \cite{schneider2022iso} 也表明,汽车网络安全已经被提
升到与功能安全同等或更重要的地位。
本文主要有以下三个贡献点:
\begin{itemize}
    \item 首先对智能网联车辆的体系结构进行了研究,并对其所面临的安全威胁进行了分析,并对其攻击来源和攻击途径进行了分析
    \item 不仅利用传统构建攻击树对智能网联汽车系统进行了威胁建模,而且还提出了一种基于微软STRIDE的改进后的攻击树威胁建模方案(称为SATT);
    \item 将SATT应用在智能网联汽车复杂的网络攻击路径中并构建模型,对构建的攻击树威胁建模,基于FAHP(模糊层次分析法)计算了攻击概率,并验证了模型的可用性。
  \end{itemize}
本文旨在通过对智能网联汽车威胁建模以及安全风险评估的研究,使得更多学者关注该领域,以此希望对智能网联汽车威胁安全建模和安全风险评估领域的发展起到推动作用。
关于本文章节安排, 首先,介绍了智能网联汽车的安全架构并提出了一种分层的车载网络模型, 
并对智能网联汽车网联系统进行了介绍,将其划分为 TSP 云端通信、V2X 通信、车载通信三个部 分进行详述,
阐述了车内外的通信原理和机制。
接着对智能网联汽车所面临的七大安全威胁进行了分析和整理,然后将攻击入口划分为物理攻击、近距离无线攻击、远距离无线攻击三个方面分布进行了详述,
并对智能网联汽车的攻击路径进行了分析。紧接着介绍了安全威胁的主流模型。威胁模型分析有助于汽车网络安全早期概念阶段的开发,
传统的安全威胁分析方法主要有: 微软的STRIDE威胁模型, 攻击树威胁模型 还有新型的HEAVENS 以及 EVITAS等威胁模型。 
目前主流的威胁模型是: 攻击树威胁模型和STRIDE威胁模型。
然而,基于传统攻击树的威胁分析存在主观因素多、准确率低的缺点。本文提出了SATT新型威胁建模方案。
因此,借助微软的STRIDE威胁模型,我们识别与资产相对应的威胁,并构建一个更全面的攻击树。最
后,对于攻击树的每个威胁攻击序列,基于 FAHP 计算攻击概率。此外,将 FAHP 与层次分析法和
传统方法进行了比较。最后对一辆自动驾驶实验车进行了网络攻击的路
径分析及建模实验,并对模型进行了基于该威胁模型的安全风险评估,最后对实验
评估数据进行了分析,验证了模型的可用性,取得了不错的效果。
结果表明,该方法在安全威胁分析中起到了一定的作用。

\sihao{\heiti{关键词:}} \xiaosi{智能网联汽车,风险评估,威胁建模, STRIDE, 攻击树, SATT}