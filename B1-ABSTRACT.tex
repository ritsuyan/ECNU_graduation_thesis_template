\vspace{-2.5cm}
\chapter*{\xiaosan \heiti{摘~~~~要}}
\addcontentsline{toc}{chapter}{摘要}
\hspace{-0.5cm}

随着工业智能、大数据、移动互联网、物联网、云计算等技术的飞速发展,
汽车正以一种全新的方式,从传统的封闭系统,转向智能城市,智能交通,智能通信,智能驾驶等新的概念融合的开放系统。
因此现代智能网联汽车将面临前所未有的安全威胁和攻击,
黑客或攻击者可以通过蓝牙、Wi-Fi 和蜂窝等外部接口非法访问和控制车辆。汽车网络安全问题正变得比以往任何时候都更加突出。
越来越多的国际安全标准也表明,汽车网络安全已经被提
升到与功能安全同等或更重要的地位。
本文主要有以下三个贡献点:
\begin{itemize}
    \item \textbf{对车载IVI系统进行了全面的系统安全分析,并利用传统的STRIDE模型对其进行了威胁建模,最后对几个典型的数据流进行了威胁评级。}
    \item \textbf{提出了一种基于STRIDE模型的改进后的攻击树威胁建模方案(称为SATT),将SATT应用在实际智能网联汽车网络,建立了威胁分析模型,并利用FAHP(模糊层次分析法)对攻击概率进行了计算,进行了风险评估,最后验证了模型的可用性。}
    \item \textbf{通过知名安全软件Isograph AttackTree + 进行了仿真建模实验。研究智能网联汽车的远程攻击方法,对汽车进行漏洞挖掘和渗透测试,通过蓝牙劫持信号、数据篡改抓包和车联APP破解等多种方式攻击车载系统,成功获取目标车辆敏感信息和实时位置,并实现了远程入侵和完全控制。}    
\end{itemize}

本文目的是通过对智能网联汽车威胁建模以及安全风险评估的研究,以期为更多学者对其在智能网联汽车威胁模型的应用提供有益的借鉴和帮助。

基于传统攻击树的威胁分析存在主观因素多、准确率低的缺点。本文提出了SATT新型威胁建模方案。
借助微软的STRIDE威胁模型,我们识别与资产相对应的威胁,并构建一个更全面的攻击树。对于攻击树的每个威胁攻击序列,基于FAHP计算攻击概率。此外,将其与传统方法进行了比较。结果表明,该方法在安全威胁分析中起到了一定的作用。最后,对实际汽车,通过蓝牙劫持信号、数据篡改抓包和车联APP反向编译破解等多种方
式对其实施了攻击。最后,还给出了针对攻击的防御措施。
\newline
\sihao{\heiti{关键词:}} \xiaosi{智能网联汽车,风险评估,威胁建模,STRIDE, 攻击树, SATT}