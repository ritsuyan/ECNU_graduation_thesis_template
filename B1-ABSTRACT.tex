\vspace{-2.5cm}
\chapter*{\xiaosan \heiti{摘~~~~要}}
\addcontentsline{toc}{chapter}{摘要}
\hspace{-0.5cm}
随着工业智能、大数据、移动互联网、物联网、云计算等技术的飞速发展,
汽车正以一种全新的方式,从传统的封闭系统,转向智能城市,智能交通,智能通信,智能驾驶等新的概念融合的开放系统。
因此现代智能网联汽车将面临前所未有的安全威胁和攻击,
黑客或攻击者可以通过蓝牙、Wi-Fi 和蜂窝等外部接口非法访问和控制车辆。汽车网络安全问题正变得比以往任何时候都更加突出。
越来越多的国际安全标准也表明,汽车网络安全已经被提
升到与功能安全同等或更重要的地位。
本文主要有以下三个贡献点:
\begin{itemize}
    \item \textbf{提出了一种基于微软STRIDE的改进后的攻击树威胁建模方案(称为SATT);}
    \item \textbf{将SATT应用在实际智能网联汽车复杂的网络攻击路径中并构建模型,对构建的攻击树威胁建模,基于FAHP(模糊层次分析法)计算了攻击概率,并验证了模型的可用性。}
    \item \textbf{研究智能网联汽车的远程攻击方法,对汽车进
    行漏洞挖掘和渗透测试,利用蓝牙协议分析、数据抓包和软件逆向等多种方
    式分析漏洞,并编写程序,成功获取目标车辆敏感信息,并实现了远程入侵和完全
    控制。}    
  \end{itemize}
本文旨在通过对智能网联汽车威胁建模以及安全风险评估的研究,使得更多学者关注该领域,以此希望对智能网联汽车威胁安全建模和安全风险评估领域的发展起到推动作用。
基于传统攻击树的威胁分析存在主观因素多、准确率低的缺点。本文提出了SATT新型威胁建模方案。
借助微软的STRIDE威胁模型,我们识别与资产相对应的威胁,并构建一个更全面的攻击树。对于攻击树的每个威胁攻击序列,基于FAHP计算攻击概率。此外,将FAHP与层次分析法和
传统方法进行了比较。结果表明,该方法在安全威胁分析中起到了一定的作用。最后,对实际商用的某款汽车,进行漏洞挖掘和渗透测试,成功攻破其安全系统,达到远程控制的效果,
在车主不知情的情况下,利用蓝牙协议,数据抓包,伪装请求等方式获取目标车辆的诸多敏
感信息并对其实施物理控制。之后,还给出了针对攻击的防御措施。


\sihao{\heiti{关键词:}} \xiaosi{智能网联汽车,风险评估,威胁建模, STRIDE, 攻击树, SATT}