\chapter{智能网联汽车攻击与威胁建模}
\label{ch3}

\section{智能网联汽车面临的安全威胁}
本节详细介绍了 ICV 中存在威胁的攻击面和
相应的攻击实验,ICV 中的潜在威胁如表 1 所示。
同时,针对 ICV 的攻击实验,总结了一些具有代表
性的攻击方法。
\subsection[]{ICV 中的潜在威胁}
在远距离通信中,恶意攻击者入侵汽车的方
式大致可分为 4 种:蜂窝网络、Wi-Fi、车载单元
(OBU,on board unit)/路侧单元(RSU,road side
unit)和全球定位系统(GPS,global positioning
system)。
\newline
1) 蜂窝网络
蜂窝网络解决了 ICV 远程通信的难题,也造成
了一些安全隐患。文献[7]破解了汽车固件(V850),
实现了对汽车设备(方向盘等)的远程控制。文献[11]
通过无线通信信道实现了对车辆的远程控制、位置
跟踪和通信监控。
\newline
2) Wi-Fi
入侵者利用 Wi-Fi 连接可以进行很多恶意操
作。如利用 Wi-Fi 远程访问车内网络;在信息娱乐
控制台植入恶意软件;对汽车 Wi-Fi 的网络流量进
行监控。在文献[5]和文献[12]中,腾讯科恩安全实
验室研究员远程入侵了特斯拉汽车的网关、BCM 和
自动驾驶系统。
\newline
3) OBU/RSU
OBU 和 RSU 是利用专用短程通信技术建立微
波通信链路来实现车辆识别和电子支付功能的设
备。然而,它们在为用户出行带来方便的同时,也
产生了一些安全隐患,文献[13]揭露了针对新兴互
联车辆的交通信号控制的拥塞攻击。
\newline
4) GPS
GPS 是汽车导航中不可缺少的一部分。在无人
驾驶中,GPS 导航作为汽车的“大脑”,能够为汽
车提供最佳的行驶路线,因此,保证 GPS 的安全是
无人驾驶领域的一项重要工作。文献[6]展示了使用
便携式 GPS 欺骗器篡改车辆的 GPS 路线,严重威
胁 GPS 的安全。
\section{近距离车外通信的潜在威胁}
在近距离车外通信中,恶意攻击者入侵汽车的
方式可分为蓝牙攻击和高频无线电攻击两类。
\newline
1) 蓝牙攻击
蓝牙作为一种近距离数据交换的通信方式,也
是恶意攻击者关注的一个攻击面。攻击者能够利用
蓝牙接口在汽车的信息娱乐单元上执行恶意代码,
从而实现对车辆内部网络的渗透和攻击。文献[14]
利用蓝牙漏洞,开发出一款名为“BlueBorn”的攻
击向量,实现了对 IVI 系统的控制。
\newline
2) 高频无线电攻击
随着高频无线电在 RKE、无钥匙点火等电子元
件中的应用,很多攻击者开始关注利用高频无线电
实现欺骗攻击的方法。文献[15-16]通过软件无线电
欺骗实现了对 RKE 系统的攻击,文献[17]则使用软
件无线电欺骗实现了对汽车 TPMS 系统的攻击。

\section{车辆内部网络的潜在威胁}
车辆内部网络的攻击面大致可分为 USB 接口
和 CAN 接口两部分。
\newline
1) USB 接口
在 ICV 中,USB 接口可以直接与 IVI 连接,实
现自动播放音频和视频文件的功能。因此,攻击者
可以在网约车、出租车等平台以播放音乐为借口,
悄悄向车内植入木马病毒从而实现对汽车的控制。
2015 年,黑客曾利用 USB 攻击造成马自达汽车 IVI
系统瘫痪。
\newline
2) CAN 接口
CAN 总线是 ECU 之间信息传输的通道,ECU
和 CAN 总线协同工作可以监控车辆状态和车辆行
为,然而,CAN 总线具有一定的脆弱性。目前,许
多汽车都安装有辅助设备(如保险狗、Mobileye、
ELM327 等),它们具有为用户提供车道偏离警告、
前方碰撞警告和车速预警等功能。攻击者可以利用
这些辅助设备的脆弱性,通过 Wi-Fi 发送控制指令,
这些设备能够将指令传输到 CAN 总线,从而使攻
击者实现对车辆状态的远程控制。文献[4]利用侧信
道攻击,通过收集 CAN 总线的数据流量窃取驾驶
员的隐私信息,验证了 ICV 中的用户隐私信息存在
被泄露的风险。

\section{ICV 中典型的攻击方法}



\section{威胁建模方法论}
威胁建模被定义为根据业务和技术利益相关者的输入,主动识别和解决对组织系统的潜在威胁的过程。通常在设计产品或新功能时完成,以避免将来出现安全漏洞的成本。

威胁建模是分析系统的各种业务和技术要求、识别潜在威胁并记录这些威胁对系统的脆弱程度的过程。威胁是指未经授权的一方访问组织的敏感信息、应用程序或网络的任何情况。 威胁建模过程的目的是清楚地了解组织的各种资产、对这些资产的可能威胁,以及如何以及何时可以减轻这些威胁。威胁建模的最终产品是一个强大的安全系统。 2020 年 4 月,视频通讯应用 Zoom 的股价从 159.56 美元跌至 111.41 美元。一旦用户群增加,Zoom 的许多安全漏洞就会暴露出来——其中大部分是 Zoom 没有预料到的。2020 年 7 月,Twitter 以一组具有内部系统访问权限的员工为目标而遭到黑客攻击,导致 当时通过比特币汇款的用户损失了 117,000 美元。 有了这样的安全攻击,品牌就会失去资本和信任。恶意软件攻击事件不会很快停止。Cyber​​security Ventures 预测,到 2021 年,网络犯罪损失每年将给全世界造成约 6 万亿美元的损失。这正是威胁建模过程可以在很大程度上减轻这些风险的地方。 
\newline
识别用于在您的应用程序中存储用户密码的过时加密算法是威胁建模的一个示例。 
\newline
漏洞是MD5等过时的加密算法。
\newline
威胁是使用暴力破解散列密码。
\newline
攻击者是试图在线出售个人信息的黑客。
\newline
缓解策略是将加密算法更改为更现代和更强大的东西。
\newline
威胁建模可以通过三种不同的方式进行: 
\newline
以资产为中心:盘点各种资产,分析每个资产的脆弱性。
\newline 
以攻击者为中心:考虑可能的攻击者、每个人想要攻击的资产以及如何攻击。
\newline
以软件为中心:关注系统设计、数据如何在各个层之间流动以及如何配置
\section{威胁建模步骤}
要进行有效的威胁建模,需要以下利益相关者的意见:
\newline
提供应用程序的业务影响的业务利益相关者。
\newline
架构师提供应用生态系统的概述。
\newline
用于特定代码输入的程序员,例如使用的框架、编码指南等。
\newline
DevOps提供服务器和网络配置的详细信息。
\newline
资源管理项目经理。

\subsection[]{设定目标}
设定目标时要牢记您的应用程序必须具有:
保护数据免受未经授权的披露的机密性
防止未经授权的信息更改的完整性
即使系统受到攻击也能提供所需的服务
在可用性和性能方面记下您承诺的 SLA。您需要保护哪些商业秘密和知识产权?也许在这个阶段最重要的问题是你想在威胁建模上花费多少时间和金钱?
\subsection[]{可视化}
这是记录组成系统的不同组件的步骤。对整个应用程序的清晰记录的概述将大大简化流程。这包括记下用例、数据流、数据模式和部署图。  您可以构建两种类型的可视化。
数据流图:它描述了数据是如何设计为在您的系统中移动的。它显示了操作级别,并清楚地显示了数据进入和退出每个组件的位置、数据存储、流程、交互和信任边界。 
流程图:它描述了用户如何在各种用例中交互和移动。它处于应用程序级别。DFD 专注于系统内部的工作方式,而 PFD 则专注于用户和第三方与系统的交互。您可以选择其中之一或同时使用两者。
现在您已经确定了应用程序中最重要的参与者和资产,是时候进行威胁评估了。
\subsection[]{识别威胁}
在上一步中,您构建了图表以了解您的系统。在此步骤中,您将需要分析这些图表以了解实际威胁。在这个阶段,您需要弄清楚您的资产可能被破坏的各种方式以及潜在的攻击者是谁。有很多方法可以做到这一点。我们将在下一节介绍六种最突出的威胁评估建模方法。 
\subsection[]{缓解}
识别完威胁后,您将获得与每个资产及其操作相关的威胁的主列表或库以及可能的攻击者配置文件列表。现在您需要弄清楚您的应用程序容易受到哪些威胁。  让我们考虑一下本文第一部分中的前一个示例。您将观察到“使用暴力破解密码”是威胁,而“使用 MD5 算法存储密码”是系统漏洞。确定漏洞后,您需要分析与每个漏洞相关的风险。基于此风险分析,您可以通过以下方式处理漏洞: 
\newline
不要做任何事情(风险太低或太难造成相关威胁)
删除与其关联的功能
关闭功能或减少功能
引入代码、基础设施或设计修复
\newline
您还将创建漏洞日志,以便在未来的迭代中随后解决。
\subsection[]{验证}
在验证期间,您检查是否所有漏洞都已得到解决。所有的威胁都被缓解了吗?是否清楚地记录了剩余风险?完成此操作后,您需要决定管理已识别威胁的后续步骤,并决定下一次威胁建模迭代的时间。请记住,威胁建模不是一次性活动。它需要在预定的时间间隔或在应用程序开发的特定里程碑期间重复。
\section{STRIDE建模方法}

常见的威胁建模方法有:基于攻击树模型的威胁建模和 STRIDE 威胁建模。攻
击树模型是 Schneier Bruce 在上世纪末提出的一种威胁建模方法 [47],其年代久远,
理论也日益完善。此方法使用树形结构搭建攻击模型,让建模人员从面临黑客攻击
的角度考虑问题,树形结构的每一个节点,都必须被慎重考虑和布局,因为每个节
点的配置稍有不慎,都有可能被黑客攻击。攻击树模型的优点是可以利用简单的网
络模型构建复杂的威胁类型和攻击方式,其扩展性强。这样的结构,还可以从深度
优先和广度优先不同的策略来考虑问题。当整个攻击树模型足够完整时,就可以很
好的预防威胁,抵御非法攻击。不过,其劣势也非常明显,由于攻击树模型完全是
从黑客的角度来思考问题的,因而,建模者必须要具备很强的技术能力,并且有较
好的攻击经验,所以难易大规模实施基于攻击树模型的威胁建模。因此,实际中,
我们更多的使用的是微软提出的 STRIDE 威胁建模方法。
\newline
STRIDE是微软开发的用于威胁建模的方法和工具。

STRIDE威胁建模的总体流程:

\subsection[]{六类威胁}

STRIDE是从攻击者的角度,把威胁划分成6个类别,分别是Spooling(仿冒)、Tampering(篡改)、Repudiation(抵赖)、InformationDisclosure(信息泄露)、Dos(拒绝服务)和Elevation of privilege (权限提升)。

\subsection[]{四类元素}

我们在来了解下四类元素,STRIDE威胁建模的第一步就是绘制数据流图,数据流图是由【外部实体】、【处理过程】、【数据存储】、【数据流】这四类元素组成。STRIDE威胁建模的核心就是使用这四类元素绘制数据流图,然后分析每个元素可能面临的上述六类威胁,针对这些威胁制定消减方法。

四类元素的介绍如下:

1.  外部实体

系统控制范围之外的用户、软件系统或者设备。作为一个系统或产品的输入或输出。在数据流图中用矩形表示外部实体。

2.  处理过程

表示一个任务、一个执行过程,一定有数据流入和流出。在数据流图中用圆形表示。

3.  数据存储

存储数据的内部实体,如数据库、消息队列、文件等。用中间带标签的两条平行线表示。

4.  数据流

外部实体与进程、进程与进程或者进程与数据存储之间的交互,表示数据的流转。在数据流图中用箭头表示。

使用以上四个元素绘制完数据流图后,还需要引入信任边界,安全的本质就是信任问题,信任边界往往就是攻击发起的地方。在数据流图中可以用红色的虚线隔离出信任边界。

如下是一个比较简单的数据流图演示:

\subsection[]{STRIDE四类元素与六类威胁的对应关系}

具体的对应关系如下图所示,并不是每个元素都会面临6个威胁,比如外部实体只有仿冒和抵赖两类威胁,我们不用关心外部实体会不会被篡改、会不会发生信息泄露、以及拒绝服务等,因为外部实体本来就是我们控制范围之外的。

其中进程(处理过程)会面临全部的6个威胁,数据存储中Repudiation(抵赖)是红色,表示只有存储的数据是审计类日志才会有抵赖的风险,存储其他数据的时候无抵赖。

\subsection[]{威胁建模的整体流程}



