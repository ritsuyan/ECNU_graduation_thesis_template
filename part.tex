\section{智能网联汽车主要攻击手段}
发明车载网络协议时,安全问题并不是主要问题。因此,许多安全功能天生就缺失了。例如,CAN 缺乏必要的保护来确保信号的可用性、机密性和真实性\cite{woo2014practical}。
FlexRay 虽然能够在出现错误的情况下保持正确操作,但无法抵御格式良好的恶意错误消息\cite{kleberger2011security}。尽管如此,这些缺点在过去并未构成迫在眉睫的安全威胁,因为车辆很少与外界连接,
而老式的安全攻击通常需要对车载网络进行物理访问。

然而,现代车辆正在通过各种方式迅速变得更加互联,用于许多高级应用。例如,车辆可以通过 DSRC(专用短程通信)连接以实现 VANET(车载自组织网络)功能,
通过 Wi-Fi/蓝牙实现车载娱乐,并通过蜂窝网络实现远程信息处理服务。尽管这些连接使车辆更加智能和舒适,但它们也将车载网络大量暴露给外部对手。
例如,CAN 通信可能会被智能手机恶意软件通过蜂窝网络远程篡改\cite{woo2014practical}。软件病毒可能通过受感染的娱乐媒体(如 CD(光盘)或蓝牙播放器)传播到车载组件。
此外,还可以通过攻击OEM存储中心的ECU密钥管理不善来侵入车载网络。

此外,直接访问的威胁仍然存在,因为攻击者也可能物理侵入通信线路,直接针对网络组件的弱点发起攻击。这种典型的攻击可能包括反汇编可执行代码和将恶意代码注入运行时环境。

车载网络的安全漏洞不仅可能对车辆用户造成严重后果,还会对其他道路交通参与者造成严重后果。例如,安全漏洞可能导致车辆用户的隐私泄露。目标私人数据可能包括车辆诊断流
、机柜对话、摄像记录、驾驶模式和车辆位置\cite{amoozadeh2015security}. 
这种典型的攻击是通过未经授权的窃听进行的。其次,安全漏洞可能导致车主或原始设备制造商的直接金钱损失。在此类攻击中,
攻击者经常故意修改或重放所需的车载数据以实现非法收益,例如车辆盗窃或里程表欺诈。第三级安全漏洞可能会对车辆使用者造成安全威胁。
这种攻击通常涉及对安全关键车载数据的恶意修改或伪造,例如轮胎压力、车速、发动机扭矩请求和制动命令. 
这可能导致非自愿驾驶机动甚至交通事故。考虑到自动驾驶的出现,这种危险至关重要,值得研究界更多关注。
第四,车载网络安全漏洞会对其他道路参与者造成安全威胁,甚至瘫痪整个交通系统。由于车辆将在大型网络中互连,
例如 VANET,因此信号可信度对于协调交通系统中的所有车辆都极为重要\cite{harding2014vehicle}.
 但是,如果车载网络安全受到损害,这种可信度可能会被破坏。
 例如,被篡改的车载网络可能会产生虚假数据,如果虚假数据已经传播到车辆外部并被其他人认为是“值得信赖的”,则可能对其他车辆造成极大的危险。
综合上述研究现状,将攻击手段分为以下类型:
\begin{itemize}
    \item 远距离通信攻击: 如利用蜂窝网络、Wi-Fi等进行伪装拦截通信信号等从而达到攻击的目的。
    \item 近距离车外通信: 利用蓝牙攻击和高频无线电攻击。如通过蓝牙连接车载娱乐系统,伪装发送信号给车载娱乐系统从而达到攻击的目的。
    \item 车辆内部网络: 如通过车辆内部USB攻击IVI系统等。
\end{itemize}

举例来说,在智能网联汽车中手机APP存储用户信息的过时加密算法是威胁建模的一个应用。
\begin{itemize}
    \item 漏洞是MD5等过时的加密算法。
    \item 威胁是使用暴力破解散列密码。
    \item 攻击者是试图在线出售个人信息的黑客。
    \item 缓解策略是将加密算法更改为更现代和更强大的东西。
  \end{itemize}
威胁建模可以通过三种不同的方式进行:
\begin{itemize}
  \item 以资产为中心:盘点各种资产,分析每个资产的脆弱性。
  \item 以攻击者为中心:考虑可能的攻击者、每个人想要攻击的资产以及如何攻击。
  \item 以软件为中心:关注系统设计、数据如何在各个层之间流动以及如何配置。
  \item 缓解策略是将加密算法更改为更现代和更强大的东西。
\end{itemize} 
通过威胁建模,我们能够实现以下这些价值:
\begin{itemize}
    \item 识别体系化的结构缺陷:大多数安全问题是设计缺陷问题,而不是安全性错误。威胁建模能帮助识别这些设计缺陷,从而减少风险敞口,指导安全测试,并降低因安全漏洞而造成的品牌损害或财务损失等可能性。
    \item 节约组织安全成本:通过对威胁进行建模,并在设计阶段建立安全性需求,降低安全设计缺陷导致的修复成本。在需求管理和威胁分析阶段,与业务开发团队高效互动,释放安全团队的专业能力,专注于高性价比的安全建设。
    \item 落地DevSecOps(开发、安全和运营)文化:通过威胁建模跑通开发和安全工具的流程集成,把风险管理嵌入产品的完整生命周期,从而推动形成完整的DevSecOps工具链。
    \item 满足合规要求:威胁建模是国际安全行业通用的方法论,通过向管理层和监管机构提供产品的风险管理活动的完整记录,帮助团队遵守全球法律法规要求,包括PCI DSS、GDPR、HIPAA、CSA STAR等。
  \end{itemize}


  \section{层次分析法和FAHP}
  这里我们介绍下层次分析法用于第四章进行新的安全模型的理论基础。
  AHP 是一种结合定性和定量分析的多目标决策分析方
法,由萨蒂在20世纪70年代提出\cite{saaty1990make}。这种方法的主
要思想是通过将复杂的问题分解成几个层次来比较每两
个决策元素的相对重要性,每个层次由有限数量的决策
元素组成。然后建立满足性质(2)(3)的判断矩阵。通过
计算判断矩阵的最大特征值和相应的特征向量,从两两
比较中间接评估不同元素的相对重要性。层次分析法的
特点是将人们的主观判断数学化,使决策更容易被接
受。AHP 的过程可分为以下步骤:
\begin{itemize}
  \item  构建一个问题的层次结构,包括一个目标、实现
  目标的要素以及对这些要素进行评分的评估标
  准。
  \item  将决策元素成对比较,根据标度准则确定相对
  重要性,然后构造一个判断。
  \item  计算每个元素的权重并检查判断矩阵的一致
  性。
  \item 通过比较综合重要性,根据所有备选方案的优先
  级做出最终决策。
\end{itemize}

然而,层次分析法难以检查判断矩阵的一致性,其一
致性判断准则缺乏科学依据。此外,当有许多评价指标
时,很难保证决策的一致性。在这种情况下,FAHP 应运而
生\cite{min1997fuzzy}。一种是基于模糊数的 FAHP,另一种是基于模糊判
断矩阵的。FAHP 的过程与层次分析法相同,但仍有两个
不同之处:
\begin{itemize}
  \item  建立的判断矩阵不同:层次分析法是判断矩阵,而
  FAHP 是模糊判断矩阵
  判断矩阵。
  \item  寻找矩阵中每个元素相对重要性的不同加权方
  法。
\end{itemize}